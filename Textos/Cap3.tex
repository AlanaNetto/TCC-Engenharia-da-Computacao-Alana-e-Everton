    \chapter{Especificação técnica} \label{cap:especificacao_tecnica}

\section{Análise de Contexto}

    \subsection{Visão Geral}
    O desenvolvimento deste protótipo tem a finalidade de auxiliar o professor no ensino de programação para crianças.
    
    O software, apresentado em forma de jogo com o tema reciclagem, apresentará diversos desafios em que a criança deverá solucionar criando sequências lógicas com blocos físicos. Após a ordenação dos blocos, de maneira em que achar correta, pela criança, ela deverá tirar fotos da sua solução e submetê-la para a avaliação do desafio dentro do aplicativo.
    
    Ao finalizar a captura da solução proposta, o aplicativo enviará a sequência de imagens para um servidor, hospedado na nuvem, o qual fará a análise das imagens dos blocos
    por meio de visão computacional a fim de identificar e converter os blocos em ações para o jogo.
    
    Ao final da conversão, o servidor devolverá para o jogo a solução proposta em ações.
    Caso a solução esteja correta, a criança passará para o próximo desafio, caso contrário, será oferecido uma nova tentativa.
    
    O aplicativo coletará dados durante o desafio para o preenchimento de relatórios que serão apresentados para o professor ou tutor através de um portal.
    
    A figura \ref{figura:diagrama_blocos} apresenta a visão geral do sistema proposto.
    
    \begin{figure}[h!]
        \centering
        \caption{Visão geral do sistema}
        \includegraphics[width=12cm]{images/cap3/diagrama_blocos.png}
        \caption*{Fonte:o autor (2020)}
        \label{figura:diagrama_blocos}
    \end{figure}
    
    % \subsection{Condições Restritivas}
    
        % \subsubsection{Custos}
        
        % \subsubsection{Físicas e Ambientais}
        
        % \subsubsection{Tecnológicas}
        
        % \subsubsection{Energização}
        
        % \subsubsection{Interferências devido ao meio}
    
    % \subsection{Benefícios e Impactos}
    
    %     \subsubsection{Econômicos}
        
    %     \subsubsection{Operacionais}
        
    %     \subsubsection{Estratégicos}
        
    %     \subsubsection{Políticos}
        
    %     \subsubsection{Sociais}

\section{Análise Funcional e de Requisitos Tecnológicos}

    \subsection{Lista de Funcionalidade e Atores}
    O sistema será composto  pelas seguintes funcionalidades:
    \begin{itemize}
        \item Desafios de lógica com o tema reciclagem;
        \item Identificação dos blocos;
        \item Conversão dos blocos em ações;
        \item Relatórios de jogo para acompanhamento do professor;
    \end{itemize}
    
    O sistema tem como atores a criança e professor/tutor.
    
    A criança é responsável pela interação com os blocos e aplicativo.
    O professor/tutor é responsável pela interação com os dados adquiridos durante a partida da criança.
    
    \subsection{Comunicação}
    A comunicação entre o aplicativo e o servidor ocorrerá de maneira unidirecional e utilizará arquitetura \textit{REST}, através da conexão Wi-Fi com a Internet.
    A comunicação entre o aplicativo e o servidor ocorre de maneira unidirecional. 
    O aplicativo envia os dados e imagens para o servidor. Após o servidor salvar os dados e processar as imagens, ele retorna para o aplicativo.
    
    \subsubsection{REST}
    O \textit{REST (Representational State Transfer)} é uma abstração da arquitetura da \textit{Web}. Consistem em regras que permitem a criação de projetos com interfaces bem definidas, permitindo a comunicação entre aplicações.
    
    
%     \subsection{Processamento}
    
%     \subsection{Interface Homem-Máquina}
    
%     \subsection{Sistemas Controlados Automaticamente}
    
%     \subsection{Aquisição de dados e Atuação}


\section{Análise da Arquitetura do Sistema}

    \subsection{Hardware}
    O hardware do sistema será composto por blocos físicos e o dispositivo mobile com câmera.
    
    Os blocos físicos serão construídos de PLA, impressos em 3D, com o tamanho aproximado de 10cm x 10cm x 5cm, seus cantos serão arredondados para evitar acidentes no manuseio. Para facilitar a identificação, além da sua cor, cada bloco possuirá um simbolo representado o tipo da ação.
    
    O dispositivo mobile deverá ser um celular \textit{Android} com camêra, podendo variar entre os modelos existentes no mercado.
    
    % \subsection{Software}

