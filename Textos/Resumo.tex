% %  -----------------------------------------------------------------------
% % |                       Resumo - Obrigatório                            |
% %  -----------------------------------------------------------------------
    \setlength{\absparsep}{18pt} 
    \begin{resumo}
    
A tecnologia tem ganhado cada vez mais relevância, um dos campos fortemente impactado pela tecnologia é a Educação, por conta das facilidades que a tecnologia proporciona em relação ao acesso à informação e possibilidades de novas metodologias de ensino.
Apesar desses benefícios, poucos professores no Brasil fazem uso de tecnologias em sala de aula, restringindo a utilização somente a laboratórios de informática, como diz a pesquisa do Programa Nacional de Formação Continuada em Tecnologia Educacional. Esse resultado decorre da insegurança na utilização de tecnologias devido à complexidade e altos custos das soluções atuais. É retratado, na mesma pesquisa, que a introdução de novas tecnologias nas escolas apresenta carência de investimentos.
Portanto, pode-se concluir que existe uma demanda de tecnologias acessíveis financeiramente que o profissional possa utilizar em sala de aula de forma fácil e intuitiva.
Há também uma necessidade por parte do aluno, visto que anualmente profissões estão exigindo um novo conhecimento tecnológico no mercado, como por exemplo a lógica de programação.
Este trabalho propõe um jogo, com o tema de reciclagem, no qual o aluno deverá guiar um personagem até um objetivo por meio de sequências lógicas construídas com blocos físicos. O aplicativo, por meio de visão computacional, reconhece os blocos contidos na imagem e os traduz em sequências lógicas que são reproduzidas pelo personagem no jogo. O jogo tem o objetivo de apresentar conceitos básicos de lógica de programação para crianças do ensino infantil e fundamental. O jogo foi desenvolvido em Unity e o reconhecimentos dos blocos foi desenvolvido em Python3 com o módulo OpenCV.
Após o desenvolvimento do aplicativo, foram realizados testes com crianças. Os testes mostraram que o jogo tem capacidade de ensinar conceitos básicos de programação, reter mais atenção das crianças do que métodos tradicionais e competência de ensinar sobre reciclagem.

         \textbf{Palavras-chave}: Visão computacional, Tecnologia Educacional, Jogos Educacionais, Educação.
    \end{resumo}

% %  -----------------------------------------------------------------------
% % |                       Abstract - Obrigatório                          |
% %  -----------------------------------------------------------------------
     \begin{resumo}[Abstract]
      \begin{otherlanguage*}{english}
 
Technology has gained more and more relevance, one of the fields strongly impacted by technology is Education, due to the facilities that technology provides in relation to access to information and possibilities of new teaching methodologies.
Despite these benefits, few teachers in Brazil make use of technologies in the classroom, restricting their use only to computer laboratories, as a research from the National Program for Continuing Education in Educational Technology says. This result comes from the insecurity in the use of technologies due to the complexity and high costs of current solutions. It is portrayed, in the same research, that the introduction of new technologies in schools presents a lack of investments.
Therefore, it can be concluded that there is a demand of affordable technologies that the professionals could use in the classroom easily and intuitively.
There is also a need on the part of the student, since annually professions are demanding new technological knowledge in the market, such as programming logic.
This project proposes a game, with recycling theme, in which the student must guide a character to an objective through logical sequences built with physical blocks. The application, through computer vision, recognizes the blocks contained in the image and translates them into logical sequences that are reproduced by the character in the game. The game aims to present basic concepts of programming logic for children in kindergarten and elementary school. The game was developed in Unity and the recognition of the blocks was developed in Python3 with the OpenCV module.
After developing the application, tests were carried out with children. The tests showed that the game has the ability to teach basic programming concepts, retain more attention from children than traditional methods and the competence to teach recycling.

   \textbf{Keywords}: Computer vision, Educational Technology, Educational Games, Education.
      \end{otherlanguage*}
    \end{resumo}
