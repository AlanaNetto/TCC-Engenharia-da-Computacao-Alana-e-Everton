% %  -----------------------------------------------------------------------
% % |                       Resumo - Obrigatório                            |
% %  -----------------------------------------------------------------------
    \setlength{\absparsep}{18pt} 
    \begin{resumo}
    A tecnologia tem ganhado cada vez mais relevância, um dos campos fortemente impactado pela tecnologia é a Educação, por conta das facilidades que a tecnologia proporciona em relação ao acesso à informação e possibilidades de novas metodologias de ensino.
    Apesar desses benefícios, poucos professores no Brasil fazem uso de tecnologias em sala de aula, restringindo a utilização somente a laboratórios de informática, como diz a pesquisa do Programa Nacional de Formação Continuada em Tecnologia Educacional. Esse resultado decorre da insegurança na utilização de tecnologias devido à complexidade e altos custos das soluções atuais. É retratado, na mesma pesquisa, que a introdução de novas tecnologias nas escolas apresenta carência de investimentos.
    Portanto, pode-se concluir que existe uma demanda de tecnologias acessíveis financeiramente que o profissional possa utilizar em sala de aula de forma fácil e intuitiva.
    Há também uma necessidade por parte do aluno, visto que anualmente profissões estão exigindo um novo conhecimento tecnológico no mercado, como por exemplo a lógica de programação.
    Este trabalho propõe um jogo, com o tema de reciclagem, no qual o aluno deverá guiar um personagem até um objetivo por meio de sequências lógicas construídas com blocos físicos. O aplicativo, por meio de visão computacional, reconhece os blocos contidos na imagem e os traduz em sequências lógicas que são reproduzidas pelo personagem no jogo. O jogo tem o objetivo de apresentar conceitos básicos de lógica de programação para crianças do ensino infantil e fundamental. O jogo foi desenvolvido em Unity e o reconhecimentos dos blocos foi desenvolvido em Python3 com o módulo OpenCV.
    Após o desenvolvimento do aplicativo, foram realizados testes com crianças. Os testes mostraram que o jogo têm capacidade de ensinar conceitos básicos de programação, reter mais atenção das crianças do que métodos tradicionais e competência de ensinar sobre reciclagem.  
    
    \textbf{Palavras-chave}: Reciclagem. Jogo. Visão Computacional. Programação. Ensino
    \end{resumo}

% %  -----------------------------------------------------------------------
% % |                       Abstract - Obrigatório                          |
% %  -----------------------------------------------------------------------
    \begin{resumo}[Abstract]
     \begin{otherlanguage*}{english}
      
       
      \textbf{Keywords}: Recycling. Game. Computer Vision. Programming. Education.
     \end{otherlanguage*}
    \end{resumo}
