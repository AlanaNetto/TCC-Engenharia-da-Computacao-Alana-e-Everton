% entre chaves, adicionar o nome do capítulo.
% o comando \label cria um nome para o capítulo que pode ser utilizado com referência
% cruzada ao longo do texto.






\chapter{     Introdução}\label{intro}

Atualmente é observada a tendência na digitalização dos serviços onde é cada vez mais presente a existência de lojas virtuais, sendo essas atuantes em sites web ou ainda aplicativos especializados. Com este crescimento, faz-se necessário também a ampliação dos serviços de entrega sendo como grande requisito a velocidade, o que tem refletivo diretamente na classe dos \textit{\textit{motoboys}}.


Esse reflexo é referente ao aumento na frota de \textit{\textit{motoboys}} nas cidades, principalmente pela presença dos aplicativos de \textit{delivery} existentes, isto permite ao entregador não ser mais dependente de uma empresa, tornando seu serviço mais acessível e expandindo seu mercado. Infelizmente o aumento de motocicletas nas ruas é acompanhado também do aumento de acidentes.


Segundo o Ministério da Saúde, os acidentes de trânsito ocupam o segundo lugar nas causas de mortes externas (acidentes e violências) do país, com os motociclistas na liderança por 42.8\% de vítimas fatais. Do total de motociclistas envolvidos em acidentes, mais de 28\% são de motorisas jovens entre 18 e 29 anos de idade \cite{giannini2019motoboys}.

Os aplicativos de \textit{delivery} mais populares não demonstram preocupações com a saúde dos \textit{motoboys} que o utilizam pois, segundo a legislação trabalhista (CLT), não há a combinação dos requisitos necessários para que os motoristas de aplicativos sejam considerados empregados, sendo a falta de subordinação a mais evidente.

Desse modo os aplicativos, apresentam-se apenas como meio rápido e barato de realizar entregas, reduzindo o tempo de contato e a relação entre  empresa e entregador, como é o caso de Ifood, Rappi, e Ubereats. Para tal, este trabalho propõe um aplicativo para monitorar o entregador no seu cotidiano, idôneo a avisar outros \textit{motoboys} por perto, através de notificações no dispositivo móvel, em caso de acidente ou necessidade de ajuda.




\section{Objetivo Geral}


Desenvolver um sistema emergencial para \textit{motoboys}, capaz de alertar áreas com alto risco de acidentes, aviso de pessoas acidentadas na proximidade com notificação, para o serviço emergencial, parentes, empregador ou traçando rota até a vítima.

Junto com isto, será possível apresentar os dados de acidentes em uma plataforma web contendo mapa, condições climáticas, áreas de risco e marcadores possuindo as informações sobre acidentes ocorridos mantendo sigilo da persona envolvida.

Além disto, será disponibilizado os dados históricos dos eventos ocorridos em formato tabular sendo possível aplicar filtros para melhoria da visualização dos relatórios.





\section{OBJETIVOS ESPECÍFICOS}

\begin{enumerate}
    \item Realizar uma pesquisa inicial para determinar o padrão do acelerômetro em caso de acidentes de moto e o tempo de resposta do acelerômetro presente nos \textit{smartphones}.
\item Verificar necessidade de \textit{hardware} externo caso o trigger do acelerômetro interno dos \textit{smartphones} não for suficiente, e, se for verificada esta necessidade, desenvolver este \textit{hardware}.
\item 	Desenvolver um aplicativo que engloba o cadastro do motorista, suas informações de contato, monitoramento e alerta de acidentes.
\item 	Desenvolver a funcionalidade de relatórios das áreas com maior risco de acidentes.
\item 	Desenvolver Integração do Google Maps para ajuda em criação de rota e análise dos eventos ocorridos.
\item 	Desenvolver as aplicações do servidor permitindo comunicação do app.
\item 	Desenvolver e implementar um esquema para banco de dados estruturado.
\item 	Integrar e testar as aplicações (servidor mais aplicativo).
\item 	Criar uma página web para acesso público dos dados adquiridos.
\item 	Analisar a possibilidade de integração com os aplicativos de \textit{delivery} existentes.

\end{enumerate}

