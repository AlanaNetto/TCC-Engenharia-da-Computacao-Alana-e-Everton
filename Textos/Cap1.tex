% entre chaves, adicionar o nome do capítulo.
% o comando \label cria um nome para o capítulo que pode ser utilizado com referência cruzada ao longo do texto.

\chapter{Introdução}\label{intro}

A tecnologia tem ganhado cada vez mais relevância em diversas áreas como saúde, indústria, agricultura, entre outros. Um dos campos fortemente impactado pela tecnologia é a Educação. Isso ocorre por conta das facilidades que a tecnologia proporciona em relação ao acesso à informação e pelas possibilidades de novas metodologias de ensino. 


\section{Problema}

Apesar de todos os benefícios que a tecnologia trás para a educação, poucos professores fazem uso de novas tecnologias. Conforme a pesquisa do Programa Nacional de Formação Continuada em Tecnologia Educacional \cite{suenia_andre_2012} , na maioria dos casos, a utilização da tecnologia fica restrita somente a laboratórios de informática. Esse resultado decorre da insegurança desses profissionais na utilização da tecnologia no cotidiano. Observa-se ainda, de acordo com a pesquisa, que a introdução de novas tecnologias nas escolas apresenta grande carência de investimentos.

Portanto, pode-se notar que o alto custo de  tecnologias com foco em ensino é um fator que dificulta a ampla utilização de tecnologias no ensino básico. Para que os professores possam utilizá-las de forma fácil e intuitiva nos ambientes escolares. Há também uma necessidade desse aprendizado por parte dos alunos, visto que inúmeras profissões estão exigindo um novo conhecimento tecnológico no mercado de trabalho, como a utilização da lógica de programação. Outro ponto a ser citado é a necessidade de inserção de jogos eletrônicos como agentes de ensino, devido à grande motivação e aderência dos alunos com esse tipo de abordagem \cite{kaue_tatiane_marcos_2017}.


\section{Justificativa}

Com a evolução dos perfis de trabalho, necessitando cada vez mais de conhecimentos básicos de programação, aumenta a procura de tecnologias no ambiente educacional, possibilitando a inclusão de novos métodos de ensino utilizando jogos digitais, realidade aumentada, simuladores, etc.

Para possibilitar essa ação, este trabalho propõe um jogo, cujo objetivo é auxiliar o ensino de lógica de programação utilizando jogos digitais. \cite{onu_2015}, utilizando a lógica de programação com blocos físicos.

\section{Objetivo Geral}

Desenvolver um aplicativo capaz de auxiliar o professor no ensino da lógica de programação para crianças em idade escolar, utilizando como tema a reciclagem. No jogo o usuário deverá descartar cada tipo de lixo em sua respectiva lixeira, para tal serão utilizados blocos físicos, que serão reconhecidos por meio da aplicação de visão computacional, para compor a lógica que permitirá direcionar o personagem do jogo para percorrer o caminho correto para a lixeira.

\section{Objetivos Específicos}

\begin{enumerate}
    \item Construir os blocos físicos de acordo com o padrão definido.
    \item Desenvolver o jogo preparado para a captura e envio da imagem dos blocos.
    \item Realizar a interpretação da imagem adquirida durante o jogo utilizando visão computacional.
    \item Desenvolver a integração entre as ações do jogo e o código retornado pelo servidor.
    \item Construir banco para armazenar informações do jogo e popular um \textit{dashboard} para análises estatísticas.
    \item Realizar testes com um grupo de crianças da faixa etária determinada para validação da efetividade do sistema.
\end{enumerate}
