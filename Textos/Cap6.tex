\chapter{Conclusão}\label{cap:conclusão}

Neste trabalho foi estudada a falta de recursos tecnológicos de baixo custo e fácil utilização para educação básica. Além disso, pensando nas profissões do futuro, foi estudada a importância do ensino de lógica de programação para as crianças de hoje (2020).
Pensando nesses problemas apresentados, foi proposto e desenvolvido um aplicativo que tem o objetivo de tentar ensinar programação para crianças do ensino básico e fundamental 1. 

O aplicativo é um jogo que interage com blocos físicos por meio de reconhecimento de imagem. O aplicativo consiste em um jogo com o tema de reciclagem no qual por meio de blocos físicos, com instruções de andar, virar, esperar, repetir e blocos numerados de 0 a 9, a criança deve resolver desafios de lógica proposto pelo jogo organizando esses blocos em sequências lógicas, fotografando e submento pelo aplicativo para validar se suas instruções estão corretas ou não. O aplicativo foi desenvolvido em Unity e Python3 e os blocos foram impressos por uma impressora 3D e coloridos manualmente com tinta. O resultado final do trabalho atendeu à todos os objetivos propostos desde o desenvolvimento do aplicativo jogo e dos blocos físicos até a realização de testes com crianças.

Após o término do desenvolvimento do aplicativo jogo, foram realizados testes de funcionamento e testes com crianças. O objetivo destes testes foram analisar o funcionamento e a capacidade do aplicativo jogo em reter a atenção da crianças, capacidade de ensinar conceitos básicos de programação e competência de ensinar sobre reciclagem.

Durante o desenvolvimento do aplicativo jogo, foram identificadas algumas possibilidades de otimização para este trabalho, como por exemplo a escolha das cores dos blocos. Por meio dos testes práticos, foi observado que a cor amarelo não se comporta bem na etapa de reconhecimento de imagem pois quando é exposta diretamente a luz tem seus parâmetros de RGB facilmente distorcidos.

Após a conclusão do trabalho e realização dos testes, foram identificadas possibilidades de trabalhos futuros. Como possíveis trabalhos futuros, pode-se apontar:

\begin{enumerate}
    \item Desenvolvimento do aplicativo para outros sistemas operacionais, como o iOS. 
    
    \item Desenvolvimento de novos tipos de blocos.
    
    \item Desenvolvimento de novas fases que utilizem mais blocos ou até mesmo outros tipos de blocos.
    
    \item  Refinar o algoritmo de reconhecimento dos blocos para reconhecer blocos desenvolvidos com outros materiais, desta forma, seria possível disponibilizar um pdf com instruções de montagem de blocos e desta forma a própria criança, por meio desse passo a passo, poderia construir seus blocos com papel, deixando assim a solução ainda mais econômica.

    
\end{enumerate}









