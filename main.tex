%  -----------------------------------------------------------------------
% |                Modelo de documento Latex para TCC do                  |
% |     curso de Engenharia da Computação da Universidade Positivo        |
%  -----------------------------------------------------------------------
% |     Produção: Eduardo J Alberti      Revisão: Veronica I. Quandt      |
%  -----------------------------------------------------------------------

\documentclass{TCC_UP}

% \usepackage[section]{placeins}
\usepackage{float}
\usepackage{pdfpages}
\usepackage{cleveref}
%  -----------------------------------------------------------------------
% |                 Informações para construção da Capa                   |
% |           Título, Autores, Orientador, Universidade e Ano             |
%  -----------------------------------------------------------------------

\instituicao{Universidade Positivo} 
    \local{Curitiba}

\titulo{Aplicativo para ensino de lógica de programação utilizando visão computacional}
    \tipotrabalho{Monografia}
    \data{2020}
    %\TipoPesquisa % Descomentar essa linha para projetos de Pesquisa

\autor{ Alana Mafra Netto  \\  Everton Henrique Carneiro }
    \curso{Engenharia da Computação}
    \escola{Escola Politécnica}
    
\orientador{Dra Veronica Isabela Quandt}
%\coorientador{Nome do Coorientador} % Descomentar se necessário

\preambulo{Monografia apresentada como requisito parcial de aprovação da disciplina de Trabalho de Conclusão de Curso de \imprimircurso  da Universidade Positivo}

\makeatletter
\hypersetup{pdftitle={\@title}, pdfauthor={\@author},         
            pdfsubject={\imprimirpreambulo},
            pdfkeywords={\imprimircurso}{\imprimirinstituicao}{TCC}, colorlinks=false,bookmarksdepth=4}
\makeatother
\makeindex

\begin{document}
    \pretextual
    \selectlanguage{brazil}
    \frenchspacing 

%  -----------------------------------------------------------------------
% |                 Imprime a Capa e a Folha de Rosto                     |
%  -----------------------------------------------------------------------
    \imprimircapa
    \imprimirfolhaderosto

%  -----------------------------------------------------------------------
% |                          Folha de Aprovação                           |
%  -----------------------------------------------------------------------
% | Na versão final do trabalho, a ser entregue correção de considerações |
% |  da banca de avaliação, será necessário incluir a Folha de Aprovação. |
% |   A Folha de Aprovação será cedida pela coordenação de TCC e poderá   |
% |               ser incluída por meio do comando abaixo.                |
%  -----------------------------------------------------------------------
    
    % \includepdf{Folha_de_Aprovacao.pdf}

%  -----------------------------------------------------------------------
% |                        Elementos Opcionais                            |
%  -----------------------------------------------------------------------

    %                       Dedicatoria
%\begin{dedicatoria}
%   \vspace*{\fill}
%   \centering
%   \noindent
%   \textit{Aqui deve ser inserido o texto da dedicatória.} \vspace*{\fill}
%\end{dedicatoria}

%                       Agradecimentos
%\begin{agradecimentos}
%    Aqui deve ser inserido o texto para os agradecimentos.
%\end{agradecimentos}

%                           Epígrafe
%\begin{epigrafe}
%    \vspace*{\fill}
%	\begin{flushright}
%		\textit{``Aqui deve ser inserido o texto da epígrafe.''}
%	\end{flushright}
%\end{epigrafe}

%  -----------------------------------------------------------------------
% |                              Resumo                                   |
%  -----------------------------------------------------------------------
    
    \begin{resumo}
\refthis{bibli}

\textbf{Palavras-chaves}: Lógica de Programação, Educação, Crianças, Jogo, Blocos
\end{resumo}

\begin{resumo}[Abstract]
\refthis[en]{bibli}

\textbf{Key-Words}: Programming Logic,  Education, Children, Game, Blocs
\end{resumo}

%  -----------------------------------------------------------------------
% |                         Listas Opcionais                              |
%  -----------------------------------------------------------------------
    
    %                           Lista de Figuras
    \pdfbookmark[0]{\listfigurename}{lof}
    \listoffigures*
    \cleardoublepage
    
% %                           Lista de Tabelas
%     \pdfbookmark[0]{\listtablename}{lot}
%     \listoftables*
%     \cleardoublepage
    
% %                           Lista de Quadros
% \pdfbookmark[0]{\listofquadrosname}{loq}
% \listofquadros*
% \cleardoublepage

% %                           Lista de Siglas
% % Note que esta lista deve ser criada manualmente
% \begin{siglas}
%   \item[ABNT] Associação Brasileira de Normas Técnicas
%   \item[UP] Universidade Positivo
% \end{siglas}

% %                           Lista de Símbolos
% % Note que esta lista deve ser criada manualmente
% \begin{simbolos}
%   \item[$ \Omega $] Ohm
%   \item[$ \Delta V $] Variação de tensão
% \end{simbolos}

%  -----------------------------------------------------------------------
% |                           Cria Sumário                                |
%  -----------------------------------------------------------------------
    \pdfbookmark[0]{\contentsname}{toc}
    \tableofcontents*
    \cleardoublepage
    
    \textual
    \pagestyle{simple}

%  -----------------------------------------------------------------------
% |                              Capítulos                                |
%  -----------------------------------------------------------------------
    
    % entre chaves, adicionar o nome do capítulo.
% o comando \label cria um nome para o capítulo que pode ser utilizado com referência cruzada ao longo do texto.

\chapter{Introdução}\label{intro}

A tecnologia tem ganhado cada vez mais relevância em diversas áreas como saúde, indústria, agricultura, entre outros. Um dos campos fortemente impactado pela tecnologia é a Educação. Isso ocorre por conta das facilidades que a tecnologia proporciona em relação ao acesso à informação e pelas possibilidades de novas metodologias de ensino. 


\section{Problema}

Apesar de todos os benefícios que a tecnologia trás para a educação, poucos professores fazem uso de novas tecnologias. Conforme a pesquisa do Programa Nacional de Formação Continuada em Tecnologia Educacional \cite{suenia_andre_2012} , na maioria dos casos, a utilização da tecnologia fica restrita somente a laboratórios de informática. Esse resultado decorre da insegurança desses profissionais na utilização da tecnologia no cotidiano. Observa-se ainda, de acordo com a pesquisa, que a introdução de novas tecnologias nas escolas apresenta grande carência de investimentos.

Portanto, pode-se inferir que existe uma demanda de tecnologias acessíveis financeiramente. Para que os professores possam utilizá-las de forma fácil e intuitiva nos ambientes escolares. Há também uma necessidade desse aprendizado por parte dos alunos, visto que inúmeras profissões estão exigindo um novo conhecimento tecnológico no mercado de trabalho, como a utilização da lógica de programação. Outro ponto a ser citado é a necessidade de inserção de jogos eletrônicos como agentes de ensino, devido a grande motivação e aderência dos alunos com esse tipo de abordagem \cite{kaue_tatiane_marcos_2017}.

Assim como tecnologia, qualidade de vida é um tema que vem ganhando mais relevância no nosso cotidiano. Soluções inovadoras e tecnológicas são apresentadas para tentar salvar o mundo dos problemas criados pela humanidade. A educação ambiental é responsável por formar pessoas mais preocupadas com problemas ambientais, que busquem a conservação e preservação do Meio Ambiente.

No ano de 2017, cerca de 42\% dos resíduos do Brasil não foram destinados corretamente, ou seja, por volta de 30 milhões de toneladas \cite{abrelpe_2017}. Pode-se afirmar que tal ação não é benéfica para o meio ambiente, já que esse material pode acabar em rios, gerar enchentes e causar impactos na saúde pública. Da totalidade de lixo produzido em território brasileiro, cerca de 30\% poderia ser reciclado, entretanto somente 3\% disso é realmente reciclado \cite{}. Um dos principais motivos disso é o fato do lixo orgânico e reciclável não serem descartados corretamente. Desse modo, reciclar é uma ótima alternativa para problemáticas de resíduos urbanos, impactando diretamente o meio ambiente, sociedade e economia. 

\section{Justificativa}

Com a evolução dos perfis de trabalho, necessitando cada vez mais de conhecimentos básicos de programação, aumenta a procura de tecnologias no ambiente educacional, possibilitando a inclusão de novos métodos de ensino utilizando jogos digitais, realidade aumentada, simuladores e etc.

A qualidade de vida, junto com sustentabilidade, são temas que, ano após ano, vem ganhando mais força, nos fazendo pensar e agir de forma a inclui-los no nosso cotidiano através de reciclagem, práticas para diminuir a produção de lixo e reduzir o impacto ambiental.

Com isso em vista, fica clara a necessidade de ensinar para as crianças a importância da reciclagem e principalmente o descarte correto do lixo de uma maneira simples didática, aproveitando tudo o que a tecnologia tem a oferecer.

Para possibilitar essa ação, este trabalho propõe um jogo, cujo objetivo é fazer o descarte correto de lixo, Objetivo de Desenvolvimento Sustentável da ONU número 12.5 \cite{onu_2015}, utilizando a lógica de programação com blocos físicos.

\section{Objetivo Geral}

Desenvolver um aplicativo capaz de auxiliar o professor no ensino da lógica de programação para crianças em idade escolar, utilizando como tema a reciclagem. No jogo o usuário deverá descartar cada tipo de lixo em sua respectiva lixeira, para tal serão utilizados blocos físicos, que serão reconhecidos por meio da aplicação de visão computacional, para compor a lógica que permitirá direcionar o personagem do jogo para percorrer o caminho correto para a lixeira.

\section{Objetivos Específicos}

\begin{enumerate}
\item Construir os blocos físicos de acordo com o padrão definido.
\item Desenvolver o jogo preparado para a captura e envio da imagem dos blocos.
\item Realizar a interpretação da imagem adquirida durante o jogo utilizando visão computacional.
\item Desenvolver a integração entre as ações do jogo e o código retornado pelo servidor.
\item Construir banco para armazenar informações do jogo e popular um \textit{dashboard} para análises estatísticas.
\item Realizar testes com um grupo de crianças da faixa etária determinada para validação da efetividade do sistema.
\end{enumerate}

    \chapter{Revisão Bibliográfica} \label{cap:rev}
Este capítulo tem como objetivo apresentar as bases teóricas que apóiam e norteiam o desenvolvimento do projeto e analisar outros trabalhos relacionados ao tema para ilustrar e proporcionar uma melhor compreensão para o leitor.

\section{Fundamentação teórica} 
Frequentemente são publicados novos estudos especulativos sobre as transformações que ocorrerão no mercado de trabalho no futuro. Apesar das análises e hipóteses variarem, a maioria aponta que diversas profissões de hoje ficarão obsoletas e em paralelo a isso, diversas outras novas profissões nascerão. No artigo \textit{21 Jobs Of The Future feito pela Cognizant Technology Solutions} \cite{cognizant_2017} vinte e uma novas profissões e suas principais habilidades são apontadas, servindo como um guia para conseguir um emprego ou se manter no mercado de trabalho nos próximos dez anos. Na maioria dessas novas profissões inferidas por esses e outros artigos, a habilidade de programar é aplicável direta e indiretamente. Algumas dessas profissões são: desenvolvedores de \textit{softwares}, engenheiros de \textit{machine learning}, analista de cibersegurança, engenheiros de \textit{big data}, cientista de dados, entre outras.

David Baker é um escritor, jornalista, fundador da TSOL Brasil, co-fundador da revista \textit{Wired} e um dos membros com mais antigos do corpo docente da \textit{The School of Life} de Londres. No ano de 2015 em uma palestra em São Paulo, Baker começa seu discurso com a seguinte afirmação: “O seu emprego pode não existir amanhã” \cite{carvalho_2015}. David Baker é bastante conhecido pelas suas pesquisas sobre as relação da tecnologia com o mercado de trabalho e afirma acreditar fortemente que logo grande parte das carreiras de hoje serão substituídas por robôs. Segundo Baker, não só carreiras braçais serão substituídas por robôs, mas “os engravatados também estão ameaçados”, brinca Baker.

Hoje, em 2020, já é possível ver essa migração. Indústrias de todos os tamanho, estão substituindo seus trabalhadores por robôs, um exemplo disso é a empresa de \textit{e-commerce} Amazon e sua logística interna, operada quase 100\% por tais máquinas \cite{winick_2018}. No mercado financeiro é possível observar Inteligências Artificiais atuando na compra e venda de ações de forma automatizada, com mais eficiência e assertividade do que analista financeiros. Na área da saúde, nano robôs já estão fazendo cirurgias. No setor de mobilidade urbana existem transportes completamente autônomos, como os caminhões sem motorista da empresa de transporte Uber que já transportam cargas sozinhos em rodovias americanas \cite{demartini_2016}. Até mesmo em trabalhos que exigem habilidades criativas, podemos ver robôs atuando e um exemplo disso é o comercial de uma marca de veículos de luxo escrito totalmente por uma inteligência artificial \cite{autran_2018}.

Sendo assim, torna-se evidente a importância de preparar as crianças de hoje para o mercado de trabalho do futuro. Para tal desenvolvimento, uma das habilidades mais importantes é a programação. Um exemplo dessa relevância foi a recente atualização da Base Nacional Comum Curricular (BNCC) realizada pelo Ministério da Educação (MEC). Nessa atualização, a BNCC dedica uma de suas dez competências para as tecnologias digitais no seu conceito de educação integral. Segundo a BNCC:

\begin{citacao}

Compreender, utilizar e criar tecnologias digitais de informação e comunicação de forma crítica, significativa, reflexiva e ética nas diversas práticas sociais (incluindo as escolares) para se comunicar, acessar e disseminar informações, produzir conhecimentos, resolver problemas e exercer protagonismo e autoria na vida pessoal e coletiva \cite[p. 9]{bncc_2017}.

\end{citacao}

Analisando esse trecho da Base Nacional Comum Curricular, é possível reconhecer argumentos em prol da inserção de programação na Educação Básica, como ao apontar que tecnologias digitais podem ser um excelente recurso para a comunicação de informações e resolução de problemas. A BNCC também menciona o termo “pensamento computacional” no Caderno de Matemática, comentando a importância de fluxogramas e algoritmos, que podem ser estudados nas aulas Matemática.

Em um estudo sobre ensino inicial de Programação e Robótica Educacional \cite{antonello_cardoso_2015} apontam-se que o ensino de programação pode ser interdisciplinar, ou seja, pode abranger duas ou mais áreas de conhecimento. O ensino de programação pode proporcionar interação e progresso em duas disciplinas ao mesmo tempo, desenvolvendo \textit{hard skills}, entendidas como competências técnicas. Do mesmo modo, esse ensino está diretamente ligado ao desenvolvimento de \textit{soft skills}, que são habilidades que trabalham com a relação dos indivíduos com os outros e com eles mesmos. Essas são competências como: resiliência, colaboração e comunicação, afirma o autor do livro best seller “Inteligência Emocional” \cite{goleman_2012}.

O artigo “Programar é bom para as crianças? Uma visão crítica sobre o ensino de programação nas escolas” \cite{geraldes_2014} mostra — por meio do Scratch, que é uma ferramenta de ensino de programação em blocos para crianças — que o ensino da programação desenvolve competências como criatividade e raciocínio lógico, além de estimular o aprendizado de inglês, trabalho em equipe, resolução de problemas, entre outras.

Mesmo com todos esses benefícios, ainda é pequeno o número de professores que fazem realmente o uso de tecnologias ou ensinam programação e outras tecnologias em sala de aula, os que usam, fazem somente uso de laboratórios de informática, como diz o estudo do Programa Nacional de Formação Continuada em Tecnologia Educacional \cite{suenia_andre_2012}. Esse estudo mostra que professores têm insegurança ao utilizar tecnologias em sala de aula, observa-se retratado no mesmo estudo que a inserção de novas tecnologias nas escolas sofre de falta de investimentos também. Portanto, nota-se que há uma carência de tecnologias que sejam acessíveis financeiramente e que os professores possam utilizá-las facilmente e intuitivamente em sala de aula. Também existem necessidades por parte dos alunos.

Em um artigo apresentado no Congresso Internacional de Educação e Tecnologias \cite{lima_queiroz_santana_2018}, é apresentado os seguintes estilos de aprendizagem: visual, auditivo e cinestésico. Assim, também é retratada a dificuldade do aluno de hoje em se concentrar em aulas realizadas com os métodos tradicionais e antiquados. Esse estudo ressalta a importância do uso de TIDCs (Tecnologias Digitais de Informação e Comunicação) em sala de aula, pois, segundo o estudo, TIDCs conseguem alinhar os estilos de aprendizado além de aumentar a concentração e motivação dos alunos. É por esse motivo, que na sala de aula se faz necessário observar a dinâmica do dia a dia, assim como as particularidades dos alunos, como idade, região em que vivem, interesses, etc. De acordo com o professor húngaro \cite[p. 81]{dornyei_2001} existem algumas estratégias motivacionais que demonstram efetividade no ensino, são elas: o aumento da interação dos alunos, a atribuição de tarefas interessantes e a quebra da monotonia da aprendizagem. Logo, se trabalhadas de forma interligada, essas estratégias podem tornar as aulas mais instigantes e estimular o desejo de aprender nos alunos.

Ademais, habilidades tecnológicas se tornam cada vez mais requisitadas no mercado de trabalho, reforçando assim a importância do contato do aluno com esses conceitos e ferramentas. Tendo em vista essa dificuldade de concentração dos alunos, é necessário criar estratégias que estimulem o aprendizado dos alunos. \cite{jacobsen_maffei_sperotto_2013}, sugerem o uso de jogos eletrônicos, pois por serem lúdicos, os jogos tornam a aprendizagem mais eficiente. Já que estimulam o raciocínio rápido, auxiliam na assimilação de conceitos complexos e desenvolvem a criatividade, isso porque com jogos os alunos deixam de ser ouvintes e passam a ser protagonistas do seu aprendizado. Nesse mesmo artigo, “Jogos eletrônicos: um artefato tecnológico para o ensino e para a aprendizagem”, nota-se que jogos não auxiliam apenas no aprendizado do conteúdo abordado em um determinado jogo, mas estimulam o desenvolvimento de outras competências como convivência, cooperação, troca de ideias, cumprimento de regras, entre outros hábitos de interação. Assim, acredita-se que a inserção de jogos eletrônicos pode funcionar como motivação para os alunos.

Outro ponto considerado importante no âmbito da educação é o ensino da sustentabilidade. No ano de 2016, cerca de 40\% dos resíduos sólidos não foram destinados corretamente, ou seja, por volta de 30 milhões de toneladas \cite{abrelpe_2017}. Pode-se afirmar que tal ação não é benéfica para o meio ambiente, já que esse material pode acabar em rios, gerar enchentes e causar impactos na saúde pública. Da totalidade de lixo produzido em território brasileiro, cerca de 30\% poderia ser reciclado, entretanto somente 3\% disso é realmente reciclado \cite{pnrs_2010}. Um dos principais motivos disso é o fato do lixo orgânico e reciclável não serem descartados corretamente. Desse modo, reciclar é uma ótima alternativa para amenizar a problemática de resíduos urbanos, impactando diretamente o meio ambiente, sociedade e economia.

De acordo com a “Agenda 2030 para o Desenvolvimento Sustentável”, a ONU (Organizações das Nações Unidas) tem como objetivo reduzir consideravelmente os resíduos até o ano de 2030 \cite{onu30_2015}. A ideia é fazer isso por meio prevenção, da reciclagem e do reuso. Tal ação pode ser desenvolvida por meio da Educação Ambiental. Essa estratégia possibilita a sensibilização dos alunos em relação ao meio ambiente, instruindo-os a refletir sobre a poluição e os danos causados à natureza. Sobre esse tema os PCNS (Parâmetros
Curriculares Nacionais) apontam que:

\begin{citacao}

O trabalho com o tema Meio Ambiente deve ser desenvolvido visando-se proporcionar aos alunos uma diversidade de experiências e ensinar-lhes formas de participação, para que possam ampliar a consciência sobre as questões relativas ao meio ambiente e assumirem de forma independente e autônoma atitudes e valores voltados à sua proteção e melhoria \cite[p. 46]{pcns_2001}.

\end{citacao}


Sendo assim, ao desenvolver esse tema no ambiente escolar, pode-se ensinar aos estudantes o respeito à natureza e o cuidado com o meio ambiente. Um bom jeito de trabalhar esse tema é por meio da reciclagem, ensinando as crianças sobre os benefícios que esse ato pode trazer para o meio ambiente e para eles mesmos.

Por fim, considerando os conceitos apresentados previamente é possível julgar como relevante trabalhar com o uso de programação e de jogos nas escolas. Já que essas têm capacidade de abranger diversas áreas do conhecimento, além de poder desenvolver habilidades emocionais. Assim, esse projeto visa unir o ensino de programação por meio de jogos com o tema de reciclagem, a fim de criar uma excelente ferramenta para o ensino dessas áreas nas escolas.

\section{Trabalhos Relacionados}


        \chapter{Especificação técnica} \label{cap:especificacao_tecnica}

\section{Análise de Contexto}

    \subsection{Visão Geral}
    O desenvolvimento deste protótipo tem a finalidade de auxiliar o professor no ensino de programação para crianças.
    
    O software, apresentado em forma de jogo com o tema reciclagem, apresentará diversos desafios em que a criança deverá solucionar criando sequências lógicas com blocos físicos. Após a ordenação dos blocos, de maneira em que achar correta, pela criança, ela deverá tirar fotos da sua solução e submetê-la para a avaliação do desafio dentro do aplicativo.
    
    Ao finalizar a captura da solução proposta, o aplicativo enviará a sequência de imagens para um servidor, hospedado na nuvem, o qual fará a análise das imagens dos blocos
    por meio de visão computacional a fim de identificar e converter os blocos em ações para o jogo.
    
    Ao final da conversão, o servidor devolverá para o jogo a solução proposta em ações.
    Caso a solução esteja correta, a criança passará para o próximo desafio, caso contrário, será oferecido uma nova tentativa.
    
    O aplicativo coletará dados durante o desafio para o preenchimento de relatórios que serão apresentados para o professor ou tutor através de um portal.
    
    A figura \ref{figura:diagrama_blocos} apresenta a visão geral do sistema proposto.
    
    \begin{figure}[h!]
        \centering
        \caption{Visão geral do sistema}
        \includegraphics[width=12cm]{images/cap3/diagrama_blocos.png}
        \caption*{Fonte:o autor (2020)}
        \label{figura:diagrama_blocos}
    \end{figure}
    
    \subsection{Condições Restritivas}
    O projeto proposto apresenta algumas condições restritivas, conforme descrito
    nos próximos subitens.

        \subsubsection{Custos}
        Apesar do aplicativo jogo precisar de materiais relativamente baratos para funcionar, como blocos impressos em 3D ou até mesmo papéis coloridos dobrados de formas semelhante aos blocos impressos, ainda se faz necessário o uso de um celular com sistema operacional Android com câmera para que o aplicativo funcione. 
            
        % \subsubsection{Físicas e Ambientais}
        
        \subsubsection{Tecnológicas}
        O aplicativo jogo necessita de um celular com câmera e com o sistema operacional Android com a versão igual ou superior a 5.0- Lollipop. Por se tratar de um protótipo, o aplicativo não oferece suporte para os demais dispositivos móveis com outros sistemas operacionais, como por exemplo Iphones. 
        
        \subsubsection{Energização}
        O celular é limitado em relação à energia, tendo um período máximo que uma carga pode sustentar, esse período máximo varia conforme o modelo e o uso do dispositivo. Para diminuir os efeitos causado por essa limitação, recomenda-se o uso do aplicativo com a bateria cheia ou próximo a uma tomada caso seja necessário recarregar a bateria do dispositivo móvel.   
        
        % \subsubsection{Interferências devido ao meio}
    
    \subsection{Benefícios e Impactos}
    O aplicativo jogo apresenta alguns benefícios e impactos, conforme descrito nos próximos subitens.

        \subsubsection{Econômicos}
        Além de um celular com câmera, o aplicativo jogo proposto é capaz de funcionar com recursos relativamente baratos, como blocos impressos em 3D ou até mesmo uma folha colorida dobrada em formatos semelhantes aos cubos; o aplicativo jogo também funciona de forma simples. Portanto pode ser utilizado em casa ou implantado em escolas de forma fácil e econômica para proporcionar a crianças um contato inicial com temas como lógica de programação e sustentabilidade, além de atender às novas demandas da BNCC para o ensino.
        
        % \subsubsection{Operacionais}
        
        % \subsubsection{Estratégicos}
        
        \subsubsection{Políticos}
        A mais nova atualização da Base Nacional Comum Curricular destina uma de suas dez competência a educação integral por meio de tecnologias digitais e faz uso, no caderno de matemática, do termo “pensamento computacional”. Pensando nisso, o aplicativo jogo proposto possibilita, de uma forma simples e econômica, um meio para trabalhar essas competências nas escolas. 

        \subsubsection{Sociais}
        O aplicativo jogo apresenta benefícios sociais para as crianças, pois as crianças terão acessos a conceitos básicos de lógica de programação e oportunidade de exercitar esse conceitos, o que pode auxiliar em competências como raciocínio lógico, resolução de problemas, pensamento computacional, entre outras habilidades que tem sido cada vez mais requisitadas no mercado de trabalho. Isso sendo proporcionado por um jogo, além de gerar maior engajamento no ensino de crianças, pode desenvolver  também competências como cooperação, cumprimento de regras, controle de impulsividade, auxílio na tomada de decisões, mais facilidade para lidar com erros e fracassos entre outras habilidades sociais.
        Além disso, o aplicativo jogo, por meio do tema de reciclagem, pode desenvolver senso de sustentabilidade auxiliando na compreensão da importância do descarte correto do lixo, o que impacta direta e positivamente  o meio ambiente.

\section{Análise Funcional e de Requisitos Tecnológicos}

    \subsection{Lista de Funcionalidade e Atores}
    O sistema será composto  pelas seguintes funcionalidades:
    \begin{itemize}
        \item Desafios de lógica com o tema reciclagem;
        \item Identificação dos blocos;
        \item Conversão dos blocos em ações;
        \item Relatórios de jogo para acompanhamento do professor;
    \end{itemize}
    
    O sistema tem como atores a criança e professor/tutor.
    
    A criança é responsável pela interação com os blocos e aplicativo.
    O professor/tutor é responsável pela interação com os dados adquiridos durante a partida da criança.
    
    \subsection{Comunicação}
    A comunicação entre o aplicativo e o servidor ocorrerá de maneira unidirecional e utilizará arquitetura \textit{REST}, através da conexão Wi-Fi com a Internet.
    O aplicativo envia os dados e imagens para o servidor. Após o servidor salvar os dados e processar as imagens, ele retorna para o aplicativo.
    
        \subsubsection{REST}
        O \textit{REST (Representational State Transfer)} é uma abstração da arquitetura da \textit{Web}. Consistem em regras que permitem a criação de projetos com interfaces bem definidas, permitindo a comunicação entre aplicações.
    
    
    % \subsection{Processamento}
    
%     \subsection{Interface Homem-Máquina}
    
    % \subsection{Sistemas Controlados Automaticamente}
    
%     \subsection{Aquisição de dados e Atuação}


\section{Análise da Arquitetura do Sistema}

    \subsection{Hardware}
    O hardware do sistema será composto por blocos físicos e o dispositivo mobile com câmera.
    
    Os blocos físicos serão construídos de PLA, impressos em 3D, com o tamanho aproximado de 10cm x 10cm x 5cm, seus cantos serão arredondados para evitar acidentes no manuseio. Para facilitar a identificação, além da sua cor, cada bloco possuirá um simbolo representado o tipo da ação.
    
    O dispositivo mobile deverá ser um celular \textit{Android} com camêra, podendo variar entre os modelos existentes no mercado.
    
    % \subsection{Software}

    \chapter{Desenvolvimento} \label{cap:desenvolvimento}

\section{Hardware}

    Para a impressão dos blocos foi utilizado a impressora 3D SnapMaker com o filamento branco no material PETG. Foi impresso uma peça nas medidas definidas no capitulo 3 para testes de material e tamanho conforme apresentado na Figura \ref{figura:teste_bloco}.
    
    \begin{figure}[H]
        \caption{Teste de tamanho e material do bloco físico}
        \centering
        \includegraphics[width=\linewidth]{Imagens/Cap4/teste_bloco.jpeg}
        \legend{\small{Fonte: o autor (2020)}}
        \label{figura:teste_bloco}
    \end{figure}
    
    Nos testes, verificamos que o material escolhido estava dentro dos padrões aceitáveis, mantendo a peça conforme o planejado. Analisamos o tamanho da peça impressa e identificamos que estava muito grande, podendo dificultar a captura da solução pela criança. O tamanho do bloco foi reduzido para 7cm x 7cm x 5mm e um novo teste foi realizado, obtendo sucesso.
    
    Para facilitar a identificação dos blocos numéricos, alteramos o encaixe  do bloco de loop para que o bloco numérico possa encaixar lateralmente, deixando todos os blocos com encaixes laterais conforme apresentado na Figura \ref{figura:alteracao_bloco_numerico}.
    
    \begin{figure}[H]
        \caption{Alteração do bloco numérico}
        \centering
        \includegraphics[width=\linewidth]{Imagens/Cap4/alteracao_bloco_numerico.jpg}
        \legend{\small{Fonte: o autor (2020)}}
        \label{figura:alteracao_bloco_numerico}
    \end{figure}
    
    As peças, impressas na cor branca, foram coloridas utilizando tinta de tecido nas cores verde, amarelo, laranja e roxo conforme apresentado na Figura \ref{figura:blocos_pintados}
    
    \begin{figure}[H]
        \caption{Alteração do bloco numérico}
        \centering
        \includegraphics[width=\linewidth]{Imagens/Cap4/blocos_pintados.jpeg}
        \legend{\small{Fonte: o autor (2020)}}
        \label{figura:blocos_pintados}
    \end{figure}

\section{Software}

Para apresentar todos os softwares que compõem o aplicativo, esta seção é dividida em três subseções: jogo, software de reconhecimento dos blocos e por fim integração dos softwares.

    \subsection{Jogo}
    
    Para o desenvolvimento do jogo foi utilizado a engine Unity 3D, assets gratuitos obtidos no site Flaticon e o editor de código Visual Studio Code, a linguagem de programação utilizada foi C\# (CSharp), o jogo foi desenvolvido para smartphones com a plataforma Android 5.0 ou superior, sendo necessário acesso a internet para que o mesmo funcione.
    
    Ao iniciar o aplicativo, é apresentado para a criança a tela inicial. O fluxograma apresentado na Figura \ref{figura:fluxo_telas} mostra o fluxo de telas do menu apresentado.
    
    \begin{figure}[H]
        \caption{Fluxo de Telas do menu}
        \centering
        \includegraphics[width=10cm]{Imagens/Cap4/fluxo_menu.jpg}
        \legend{\small{Fonte: o autor (2020)}}
        \label{figura:fluxo_telas}
    \end{figure}
    
    Na tela inicial, apresentada na Figura \ref{figura:menu_final}, a criança pode ser direcionada para a tela de créditos e desafios, resolvidos e disponíveis. 
    
    \begin{figure}[H]
        \caption{Menu do Jogo}
        \centering
        \includegraphics[width=14cm]{Imagens/Cap4/menu_final.png}
        \legend{\small{Fonte: o autor (2020)}}
        \label{figura:menu_final}
    \end{figure}
    
    Ao jogar pela primeira vez é solicitado duas informações, nome e idade, para identificar a criança, conforme apresentado na Figura \ref{figura:tela_final_cadastro}. Essas informações são salvas localmente no smartphone e serão enviadas para o servidor junto com o envio de cada solução dos desafios, todas as soluções, corretas ou não, serão salvas no banco de dados para análise estatística e desempenho de cada criança.
    
    \begin{figure}[H]
        \caption{Tela de Cadastro}
        \centering
        \includegraphics[width=14cm]{Imagens/Cap4/tela_final_cadastro.png}
        \legend{\small{Fonte: o autor (2020)}}
        \label{figura:tela_final_cadastro}
    \end{figure}
    
    O jogo possui 4 fases, cada uma delas apresenta um novo bloco lógico juntamente com o descarte correto de um tipo de lixo. Para a estruturação delas foi utilizado o componente Grid da Unity. Todas as fases possuem dois grids, um para o chão e outro para a decoração e cada um possui seu próprio \textit{titlemap}. Os objetos das fases como lixos, lixeiras e personagem são tratados individualmente.
    
    \subsubsection{Objetos e Controladores}
    
    O jogo possui dois objetos bases, o lixo e a lixeira. O lixo possui 4 parâmetros apresentados na Figura \ref{figura:parametros_lixo}. O parâmetro \textit{Trash Type} representa o tipo do Lixo (Papel, Plástico, Metal e Vidro), o parâmetro \textit{Collected} representa se o lixo foi coletado no cenário, o \textit{Discarted} representa se o lixo foi descartado e o \textit{Discarted Correctly} representa se o lixo foi descartado na lixeira correta. 
    
    \begin{figure}[H]
        \caption{Parâmetros do Objeto Lixo Papel}
        \centering
        \includegraphics[width=10cm]{Imagens/Cap4/parametros_lixo.png}
        \legend{\small{Fonte: o autor (2020)}}
        \label{figura:parametros_lixo}
    \end{figure}
    
    A lixeira possui apenas um parâmetro conforme apresentado na Figura \ref{figura:parametro_lixeira}, o \textit{Trash Type Accepted} que representa qual tipo de lixo aquela lixeira aceita.
    
    \begin{figure}[H]
        \caption{Parâmetros do Objeto Lixeira Metal}
        \centering
        \includegraphics[width=10cm]{Imagens/Cap4/parametro_lixeira.png}
        \legend{\small{Fonte: o autor (2020)}}
        \label{figura:parametro_lixeira}
    \end{figure}
    
    O gerenciamento da câmera do dispositivo é feita através de um controlador, ele é responsável por controlar as funções básicas da câmera (abrir, fechar e tirar foto) e inciar o processo de envio da imagem para que ela possa ser interpretada pelo software de reconhecimento.
    
    Cada fase do jogo, possui uma quantidade especifica de objetos (Lixo e lixeiras), de acordo com seu nível de dificuldade, e um controlador, apresentado na Figura \ref{figura:controlador_fases}, responsável por gerenciar o envio, compilação e execução da solução, além de identificar quando o a solução proposta está correta ou não. 
    
    \begin{figure}[H]
        \caption{Controlador das Fases}
        \centering
        \includegraphics[width=10cm]{Imagens/Cap4/controlador_fases.png}
        \legend{\small{Fonte: o autor (2020)}}
        \label{figura:controlador_fases}
    \end{figure}
    
    Após receber o retorno do software de reconhecimento, a lista com os blocos passa por um compilador que verifica se a solução enviada foi estruturada corretamente, esse processo inclui a verificação da quantidade de loops e blocos numéricos presentes na solução. Após a conclusão, bem sucedida, do processo de compilação, é iniciado o processo de execução e logo em seguida a identificação se a solução enviada atingiu os objetivos da fase. Para que a identificação seja possível, o controlador possui uma lista dos lixos dispostos no cenário e valida se todos os lixos foram coletados e descartados corretamente.
    
    
    \subsection{Software de reconhecimento dos blocos}
    
    O software de reconhecimento dos blocos é uma API, isto é, uma interface de programação de aplicações,  desenvolvida em Python. Os principais módulos utilizados no desenvolvimento deste software são \textit{OpenCV, numpy, pytesseract e flask}.

    Com base no referencial teórico deste trabalho sobre visão computacional, foi desenvolvida uma API capaz de receber uma imagem dos blocos, reconhecê-los com base nas classes de blocos propostos e por fim retornar os blocos contidos na imagem submetida no formato JSON. O software de reconhecimento é limitado às seguintes restrições:
    
        \begin{itemize}
        \item Os blocos a serem reconhecidos devem exclusivamente ser desenvolvidos pelos pesquisadores e desenvolvedores deste trabalho, qualquer cópia poderá não funcionar como proposto;
        \item A imagem deve estar iluminada adequadamente (sem sombras);
        \item As cores do plano no qual os blocos estiverem dispostos devem ser diferentes das cores de qualquer um dos blocos (preferencialmente branco);
        \item A imagem deve ser capturada de forma perpendicular à superfície na qual os blocos estejam dispostos;
       \item Os blocos a serem reconhecidos devem estar centralizados na imagem e não podem estar cortados;
      \item Os blocos a serem reconhecidos não podem estar sobrepostos.
    \end{itemize}

    As etapas do desenvolvimento do algoritmo de reconhecimento dos blocos estão ilustradas no fluxograma na Figura \ref{figura:fluxo}. 
    Nas subseções subsequentes está descrito, com exemplos, o funcionamento de cada etapa do software de reconhecimento dos blocos.
    
    \begin{figure}[H]
        \caption{Fluxograma de Reconhecimento dos Blocos}
        \centering
        \includegraphics[width=\linewidth]{Imagens/Cap4/fluxo.PNG}
        \legend{\small{Fonte: o autor (2020)}}
        \label{figura:fluxo}
    \end{figure}


    \subsubsection{Aquisição}
    
    Após o recebimento da imagem através do protocolo http POST, é necessário converter a imagem de bytes para um vetor numpy através do comando \textit{fromstring} do módulo numpy, pois é neste formato que o módulo OpenCV é capaz de ler as imagens. 
    
    Uma etapa comum em visão computacional é o redimensionamento  das imagens, porém neste caso não se fez necessário o redimensionamento pois as imagens já são enviadas em um tamanho pré definido pelo aplicativo jogo.

    \subsubsection{Conversão de BGR para HSV}
    O padrão HSV consiste em cor, retratado como \textit{Hue}; a saturação da cor por \textit{Saturation} e por fim o brilho, chamado de \textit{value}, por conta disso possui o nome HSV.
    
    A etapa de conversão de BGR (\textit{blue}, \textit{green} e \textit{red}) para HSV é fundamental, pois é por meio deste padrão que as máscaras, utilizadas para isolar cada cor da imagem é criada. Portanto é por conta desta etapa que a segmentação por cores se torna possível.

    
    \subsubsection{Segmentação Por Cor}
    Nesta etapa a imagem inicial se desfaz em N imagens, onde N  é o número de blocos distintos na imagem submetida para reconhecimento dos blocos.
    
    Nesta etapa o algoritmo compara a imagem submetida para o reconhecimento com uma máscara e realiza uma filtragem de cores. Estas máscaras são criadas a partir de 3 parâmetros, o máximo de valores BGR para uma determinada cor, o mínimo valores BGR para a mesma cor, e por fim com os valores da imagem gerada através da conversão BGR para HSV da etapa anterior. 
    
    No momento em que a imagem passa por este filtro de segmentação de cor, as componentes de cor que não fazem parte do espectro definido como máximo e mínimo são zeradas, resultando em uma imagem com apenas a cor que faz parte do espectro especificado, como ilustrado nas Figuras \ref{figura:ex1_original} e \ref{figura:ex1_tratado}, nas quais, pode-se ver o antes e depois da segmantação da cor laranja de uma imagem com diversos blocos de cores distintas.
    
    \begin{figure}[H]
        \caption{Exemplo de segmentação por cor - original}
        \centering
        \includegraphics[width=\linewidth]{Imagens/Cap4/ex1_original.PNG}
        \legend{\small{Fonte: o autor (2020)}}
        \label{figura:ex1_original}
    \end{figure}
    
    
    \begin{figure}[H]
        \caption{Exemplo de segmentação por cor - após tratamento}
        \centering
        \includegraphics[width=\linewidth]{Imagens/Cap4/ex1_tratado.PNG}
        \legend{\small{Fonte: o autor (2020)}}
        \label{figura:ex1_tratado}
    \end{figure}
    
    \subsubsection{Detecção de Contorno}
    Nesta etapa, espera-se que as imagens estejam com as cores isoladas para que passem pelo processo de detecção de contorno da melhor maneira possível. Esta etapa consiste basicamente na detecção de bordas das imagens provenientes da etapa anterior.

    Através das máscaras HSV geradas, é realizada uma busca por contornos das cores que fazem parte do espectro determinado para cada cor.
    
    Depois que os contornos são encontrados, verifica-se se algum contorno possui área maior que um valor X, o qual varia dependendo do bloco, como ilustrado na Tabela \ref{table:area_values}. Se a área do contorno for maior que o valor pré determinado, o algoritmo assume, considerando a limpeza de ruídos feita na etapa de segmentação de cores e o tamanho da área encontrado nesta etapa, que aquela seção da imagem corresponde a um bloco da cor da máscara testada. Portanto por meio desta etapa, já é possível reconhecer os blocos que possuem somente uma cor para um símbolo, isto é, laranja - esperar, amarelo - caminhar, azul - virar e roxo - repetir.
    
    \begin{table}[H]
        \centering
        \caption{Tabela de valor mínimo de área}
        \label{table:area_values}
        \begin{tabular}{ |c|c| } 
         \hline
        Cor     & Mínimo valor de área \\
         \hline
        Azul    & 10000                \\
         \hline
        Amarelo & 7500                 \\
         \hline
        Laranja & 10000                \\
         \hline
        Roxo    & 12500                \\
         \hline
        Verde   & 10000                \\    [0.5ex]    
         \hline
        
        \end{tabular}
        \legend{\small{Fonte: o autor (2020)}}
        
    \end{table}
    
        
    \subsubsection{Reconhecimento Numérico}
    Por fim, para o reconhecimento dos blocos numéricos se faz necessário uma etapa de OCR, ou seja, reconhecimento ótico de caracteres, pois não são possíveis descrevê-los totalmente pelas etapas anteriores, pois além da cor verde, também é preciso determinar seu símbolo.
    
    Antes da etapa de OCR, a imagem passa por uma função de corte, na qual a região do símbolo a ser reconhecido é isolado com objetivo de diminuir possíveis erros da etapa de OCR. Portanto, se um bloco foi identificado como verde, obtém-se, por meio da função de achar contornos,  as coordenadas da região a ser reconhecida pela etapa de OCR e, por meio de suas coordenadas, essa região é cortada da imagem original como ilustrado nas Figuras \ref{figura:ex2_original} e \ref{figura:ex2_tratado}.
    
    \begin{figure}[H]
        \caption{Exemplo Tratamento para OCR - original}
        \centering
        \includegraphics[width=14cm]{Imagens/Cap4/ex2_original.PNG}
        \legend{\small{Fonte: o autor (2020)}}
        \label{figura:ex2_original}
    \end{figure}
    
    
    \begin{figure}[H]
        \caption{Exemplo Tratamento para OCR - após tratamento}
        \centering
        \includegraphics[width=7cm]{Imagens/Cap4/ex2_tratado.PNG}
        \legend{\small{Fonte: o autor (2020)}}
        \label{figura:ex2_tratado}
    \end{figure}
    
    Na etapa do OCR, a imagem cortada é confrontada com uma lista de 10 possíveis valores, números entre 0 e 9, e por meio do módulo de Python \textit{pytesseract} e do software \textit{Tesseract}, o símbolos numéricos dos blocos verdes são reconhecidos.


    \subsubsection{Envio das Informações}
    Durante as etapas anteriores, na medida em que os blocos são reconhecidos, os seus significados e, por meio da função achar contornos, suas coordenadas são armazenadas em uma estrutura de dados Python conhecida com tupla.
    
    Por fim, esta estrutura que contém os blocos e suas respectivas posições na imagem é ordenada baseada nas coordenadas do menor valor de X para o maior valor de X, em outras palavras, da esquerda para a direita da imagem. Esses valores então são  encapsulados em uma estrutura JSON e são enviados como retorno da API para o aplicativo jogo.

    \subsubsection{Testes}
    
    Os primeiros testes foram realizados no decorrer do desenvolvimento do software com uma imagem genérica de um círculo cromático. Esses primeiros testes tinham o objetivo de ver os efeitos das funções do OpenCV em uma imagem com diversas cores.
    
    Após o desenvolvimento do software, foram realizados teste com imagens dos blocos com o objetivo de ajustar os parâmetros das máscaras de segmentação por cor e os tamanhos das áreas mínimas dos contornos para se considerar um bloco. Primeiramente foram realizados testes unitários com as imagens dos protótipos dos blocos reais, cada imagem com apenas um bloco em um fundo branco como ilustrado na Figura \ref{figura:ex_teste}. Nos próximos testes foi utilizado imagens no mesmo padrão de qualidade, porém com sequências de blocos.
       
    \begin{figure}[H]
        \caption{Exemplo de imagem de teste}
        \centering
        \includegraphics[width=8cm]{Imagens/Cap3/Blocos/Virar.png}
        \legend{\small{Fonte: o autor (2020)}}
        \label{figura:ex_teste}
    \end{figure}
    
    Após o sucesso na primeira bateria de testes, foram realizados testes com imagens mais próximas da realidade, isto é, dos blocos reais em diferentes iluminações e em diferentes superfícies. Os padrões de teste foram os mesmos utilizados na primeira etapa, ou seja, em um primeiro momento os testes foram realizados com imagens com apenas um bloco para somente então ser realizado testes com imagens com sequências de blocos.
    
    Durante a etapa de testes o algoritmo foi ajustado para reduzir ao máximo erros provocados por fenômenos como sombras, iluminação, cores de superfícies e distâncias dos blocos, porém o software ainda está sujeito a erros se não utilizado com imagens em qualidade adequada, como descrito no início desta seção.    


    \subsection{Relatórios}
    
    
    \subsection {Integração dos Softwares}
    
    Para a integração dos softwares (jogo e reconhecimento dos blocos), foi utilizado a linguagem Python junto como framework Flask, transformando o software de reconhecimento em uma API com arquitetura REST, o diagrama da Figura \ref{figura:diagrama_acesso_integracao} representa o acesso ao serviço de reconhecimento.
    
    \begin{figure}[H]
        \caption{Diagrama de acesso ao serviço de reconhecimento}
        \centering
        \includegraphics[width=15cm]{Imagens/Cap4/diagrama_acesso_integracao.jpg}
        \legend{\small{Fonte: o autor (2020)}}
        \label{figura:diagrama_acesso_integracao}
    \end{figure}
    
    O serviço possui 2 rotas, sendo que, somente uma delas possui retorno gráfico ao acessar via navegador. A tabela \ref{tab:rotas} apresenta as rotas do serviço.
    
    \begin{table}[H]
        \centering
        \caption{Tabela de rotas do serviço}
        \label{tab:rotas}
        \begin{tabular}{|c|c|c|}
            \hline
            {Rota} & {Métodos} & {Parâmetros}  \\ \hline
            /solution & POST, PATCH & image, child, rightSolution \\ \hline
            /dashboard & GET & childID \\ \hline
        \end{tabular}
        \legend{\small{Fonte: o autor (2020)}}
    \end{table}
    
    A rota \textit{/solution} é responsável por gerenciar as soluções propostas. O método POST recebe a imagem da solução, através do parâmetro \textit{image}, executa as ações de reconhecimento apresentadas na seção anterior e após isso, salva no banco de dados a solução e as informações da criança, recebidas pelo parâmetro \textit{child}. O método PATCH atualiza se a solução foi ou não correta, usando o parâmetro \textit{rightSolution} em combinação com o parâmetro \textit{child}.
    
    A rota \textit{/dashboard} é responsável por apresentar o relatório para os professores/tutores. O parâmetro \textit{childID} é a identificação da criança a ser visualizada, caso esse parâmetro não seja passado, é apresentado o relatório com todas as crianças.
    
    Na arquitetura criada, a comunicação é sempre iniciada pelo smartphone com o jogo. Como o serviço é segregado do jogo, ele consegue trabalhar de forma independente, podendo se conectar a outros jogos e aplicativos, desde que, o padrão estabelecido para a comunicação seja respeitado. No jogo, não existe possibilidade de trabalhar independente, pois ele necessita do serviço para reconhecer os blocos na solução enviada, tornando obrigatório o acesso a Internet.
    
    \subsubsection{Padrão de comunicação}
    
    Para possibilitar a integração entre o jogo e o software de reconhecimento, foi definido um padrao de resposta da API. A resposta da rota \textit{/solution} para quando é submetido uma solução através do método POST contém um encapsulador, representado pela propriedade \textit{blocks}. Esse encapsulamento da resposta, possibilita a interpretação do JSON recebido na Unity 3D, visto que, para que consiga ser interpretado, todas as propriedades do JSON devem ser traduzidas para classes dentro do jogo, O conteúdo retornado nessa propriedade é uma lista com os objetos que representam os blocos, conforme apresentado na Figura \ref{figura:json_retorno}.
    
    \begin{figure}[H]
        \caption{JSON de retorno}
        \centering
        \includegraphics[width=\linewidth]{Imagens/Cap4/json_retorno.png}
        \legend{\small{Fonte: o autor (2020)}}
        \label{figura:json_retorno}
    \end{figure}
    
    O objeto que representa o bloco possui uma propriedade, \textit{name}, que contém o nome do bloco. Caso o bloco identificado seja o bloco numérico, o mesmo retorna o seu respectivo numero precedido da palavra \textit{Number}.
    Ao receber a lista de blocos, a Unity os converte em objetos, representados pela classe \textit{Block}. Quando é identificado a presença de um bloco de Loop, o mesmo é convertido para a classe \textit{Loop}. Essa classe herda a classe \textit{Block}, e possui um acréscimo de duas propriedades, \textit{repeatTimes} e \textit{blocks} para identificar quantas vezes é necessário executar os blocos dentro do loop.
    
    
    
    \chapter{Testes e Resultados}\label{cap:conclusão}

Neste capítulo será abordado os testes e resultados dos sensores, consumo de bateria e tempo de resposta do sistema embarcado e aplicativo \textit{Android}.

\section{Aplicação}

Nesta seção será discutida os valores encontrados na realização dos testes no aplicativo além de seu consumo e tempo de resposta para a notificação.

\subsection{\textbf{Sensores}}


Antes do desenvolvimento do protótipo do projeto foi desenvolvido um aplicativo para monitoramento exclusivamente dos sensores acelerômetro e giroscópio no \textit{Android}. O intuito deste aplicativo era observar o desempenho e o tempo de resposta dos sensores utilizados. Os testes foram realizados em um celular \textit{Android} 8.1 de modelo \textit{Zenfone} 4 ZE554KL,  o celular foi submetido a um teste de  8 horas para verificar o consumo da bateria e mais  um teste de 3 horas para a mensuração o comportamento dos sensores no trajeto até a faculdade.


No primeiro cenário o celular ficou com o aplicativo executando durante um período de 8 horas, nessas 8 horas  foram realizados movimentos com o aparelho e  períodos inertes. Este teste foi executado durante 3 dias para verificar seu comportamento.

No segundo cenário o aplicativo foi acionado durante o trajeto diário sendo assim o aplicativo  ficou exposto aos movimento do veículo, ao final do trajeto o aplicativo armazena todos os dados coletados e é encerrado. Este teste foi repetido durante uma semana.


Com o teste de uso do aplicativo em \textit{background} durante o período de 8 horas foi possível observar o consumo do celular chegando ao valor médio de 8 mAh  este dado de consumo foi retirado do próprio aparelho pois o sistema  o Android disponibiliza um relatório de consumo de energia de cada aplicação sendo executada no sistema. Com isto é possivel perceber um consumo muito baixo para um celular com autonomia de  3300 mAh, representando um consumo de 0,24\% da aplicação. 



No segundo teste foram coletados mais de dez mil amostras tanto para o acelerômetro quanto para o giroscópio, todos com o tempo mínimo de leitura de  100ms e sendo executados paralelamente. No entanto este tempo mínimo não se provou verdadeiro com os dados adquiridos. Realizando uma média dos valores coletados foi possivel perceber que o giroscopio possui mais que 400\% do valor minimo e o acelerometro mais que 20\%. A Tabela 2 mostra a média de todas as amostras, alem disto a Tabela 3 mostra uma amostra dos dados coletados.  




\begin{table}[]
    \centering
    \caption{Tabela com valores médios do giroscopio e acelerometro}
    \begin{tabular*}{\textwidth}{l@{\extracolsep{\fill}}cccc}
\toprule
{} &           tempo(ms) &  amostras  & tipo\\
\midrule
0 &  475.57 &  98638035 &  giroscpio \\
0 &  122.89 &  118510710 &  acelerometro \\
\bottomrule
\end{tabular*}

\end{table}



\begin{table}[]
    \centering
    \caption{Amostra dos dados coletados}

\begin{tabular*}{\textwidth}{l@{\extracolsep{\fill}}lrrrrrl}
\toprule
{} &     pk &         x &         y &         z &  tempo(ms) &          tipo \\
\midrule
0 &  11554 &  0.239120 & -0.123550 & -0.058319 &        396 &    giroscopio \\
1 &  11555 & -0.127335 & -0.287064 & -0.117981 &        322 &    giroscopio \\
2 &  11556 & -0.193375 &  0.315353 &  0.053528 &        383 &    giroscopio \\
3 &  11557 &  0.234863 & -0.219421 &  0.080704 &        517 &    giroscopio \\
4 &  11558 &  0.017014 &  0.067139 &  0.057800 &        349 &    giroscopio \\
0 &  26036 & -0.780502 &  6.909668 &  7.022186 &        100 &  acelerometro \\
1 &  26037 & -0.790085 &  7.223312 &  6.564896 &        146 &  acelerometro \\
2 &  26038 & -0.462067 &  7.565689 &  6.313507 &        192 &  acelerometro \\
3 &  26039 & -0.258560 &  7.821869 &  6.016632 &        137 &  acelerometro \\
4 &  26040 & -0.100555 &  7.596802 &  2.372650 &        136 &  acelerometro \\
\bottomrule
\end{tabular*} 
\end{table}




Com os dados adquiridos foi  possível notar que, o celular consegue coletar 10 amostras dos sensores a cada segundo em média sendo que, a cada quatro amostras coletadas pelo acelerômetro é feita a coleta de uma amostra do giroscópio. Sendo assim, é possível concluir que o giroscópio possui um tempo médio de 500ms e cerca de 125ms para cada leitura do acelerômetro. Com este estudo foi possível gerar, ao todo, 108 mil amostras coletadas em 3 horas de testes.

Com isto foram gerados os gráficos para monitorar a resposta dos sensores em cada eixo. A Figura 33 demonstra os eixos em relação a um celular .

 \begin{figure}[H]

\begin{center}
     \caption{Eixos Android}
  \includegraphics[width=70mm]{images/Cap5/eixos_anjo.png}
\end{center}
 \scriptsize Fonte: Os autores
  
\end{figure}


Com os dados adquiridos foi possível gerar gráficos individuais,  podendo visualizar o comportamento de cada eixo individualmente. A Figura 34 demonstra o gráfico em relação ao eixo X,Y,Z do acelerômetro.


 \begin{figure}[H]

\begin{center}
     \caption{Gráfico eixos do acelerometro}
  \includegraphics[width=150mm]{images/Cap5/acelerao_3_eixos.png}
\end{center}
 \scriptsize Fonte: Os autores
  
\end{figure}



O eixo Y é correspondente à aceleração, e o eixo X responsável pelo número de amostras coletadas. Com a Figura 35 é possível perceber que o celular   possui uma força média de 1G mesmo estando em movimento durante  todo o percurso, os pontos de variações são ações de pegar o celular na mão e realização de gesto ao utilizá-lo.


 \begin{figure}[H]

\begin{center}
     \caption{Gráfico da aceleração dos sensores}
  \includegraphics[width=150mm]{images/Cap5/medias_.png}
\end{center}
 \scriptsize Fonte: Os autores
  
\end{figure}



Já para a o protótipo foram adotados os valores do estudo anterior como base, além de teste de quedas para verificar o comportamento dos sensores. Os testes foram realizados em etapas, o celular caiu de alturas específicas dez vezes cada, gerando valores médios dos sensores. Com os dados encontrados nos testes foi  adotada a verificação de acidentes a partir da aceleração resultante para os três eixos do acelerômetro e giroscópio. Foi escolhido utilizar a resultante de todos os eixos pois o celular poderá ficar em qualquer posição junto ao usuário. Então, com isto,  foi definido um \textit{threshold} mínimo para o reconhecimento de acidentes. A Tabela 4 demonstra os  resultados médios adquiridos durante os teste.

\begin{table}[H]
    \centering
    \caption{Valores encontrados para o acelerometro}

\begin{tabular*}{\textwidth}{l@{\extracolsep{\fill}}llllr}
\toprule
{} & Altura(m) & Velocidade Estimada (m/s) & Velocidade Estimada(Km/h) &  Velocidade (Km/h) \\
\midrule
0 &       0,5 &               3,132091953 &               11,27553103 &                                                   15,82346656 \\
1 &         1 &               4,429446918 &               15,94600891 &                                                    18,10074661 \\
2 &       1,5 &               5,424942396 &               19,52979263 &                                                     21,36943805 \\
3 &         2 &               6,264183905 &               22,55106206 &                                                    23,63484279 \\
\bottomrule
\end{tabular*}
\end{table}


Para o acelerômetro foi encontrado um \textit{threshold} de 22 km/h para a detecção de um acidente, com esta velocidade é exercida uma força gravitacional de 5,5G  podendo gerar um acidente leve.

Já para o giroscópio foi considerada a variação da angulação pois é necessário que a variação que haja um \textit{threshold} mínimo para a detecção da queda pois o usuário o irá carregar o celular mesmo estando fora do seu veículo, o próprio ato de caminhar poderia gerar um falso positivo. Com o intuito de remover a detecção de falso positivos foram realizados os seguintes testes da Tabela 5.

Os testes foram utilizados para  chegar um valor específico de \textit{threshold}, sendo assim,  o valor encontrado foi  2.20 km/h.  Esta velocidade mínima do giroscópio é utilizada para remover os falsos positivos sendo acionado somente em batidas leves ou superiores  sendo necessário que ambos os sensores estejam dentro no valor mínimo pré definido em um período de 30 amostras coletadas pelo celular.

\begin{table}[H]
\caption{Tabela de Valores encotrados para o Giroscopio}
\begin{adjustbox}{width=\columnwidth,center}

\begin{tabular}{llrll}
\toprule
{} &  Tipo &  Angulo em  (1 sec) & Velocidade Calculada(m/s) & Velocidade Calculada(Km/h) \\
\midrule
0 &  superfície macia &                   0 &             0,03213005852 &               0,1156682107 \\
1 &  superfície macia &                  90 &              0,7117882769 &                2,562437797 \\
2 &  superfície macia &                 180 &              0,9818256226 &                3,534572241 \\
3 &  superfície macia &                 360 &               1,249420492 &                4,497913771 \\
\bottomrule
\end{tabular}

\end{adjustbox}
\end{table}





\subsection{\textbf{Tempo de Resposta Firebase}}

Para o tempo  de resposta do servidor foram realizados 6 testes com internet móvel e rede \textit{Wifi}, com  e sem  sistema embarcado gerando um acidente. Com este  teste foi possível medir a velocidade de resposta entre eles e verificar qual dos sistemas responde mais rápido.O teste consistiu em utilizar dois celulares e um embarcado, um sendo responsável por receber a notificação e outro por enviar.

O tempo de resposta foi medido a  partir do momento que o celular realiza o envio da  notificação para o servidor até o recebimento da mensagem no outro celular, este intervalo de tempo  para o recebimento da notificação contempla toda a execução do sistema, partindo da primeira requisição sendo coletada  pelo servidor para as realizações de consultas no banco transacional passando para a seleção dos usuários próximos e finalizando na criação do tópico do FCM e realizando o envio para os servidores da \textit{Google}.

Com os valores adquiridos pelos teste foi concluído  que o tempo de resposta utilizando o sistema embarcado é menor, pois assim que o embarcado se comunica com o aplicativo não existe muitos filtros para a autenticação do acidente a não ser por uma validação de sequências de caracteres que definem acidente.Enquanto para o  acionamento do celular está dependente  da variação dos dois sensores sensores e do tempo de aquisição de cada leitura além de que ambos os sensores devem ser acionados em um curto período de tempo para que seja detectado o acidente. A Tabela 6 compara os tempos encontrados.



\begin{table}[H]
    
    \caption{Tabela valores médios do giroscopio e acelerometro}
    \begin{tabular*}{\textwidth}{l@{\extracolsep{\fill}}lllllll}
\toprule
                Tipo &       Conexão & Tempo  1 (s) & Tempo  2 (s) & Tempo  3 (s) \\
\midrule
           Android &  Dados móveis &                    3,57 &                    3,49 &                    3,57  \\
            Android &         WI-FI &                    2,79 &                    2,65 &                     2,7  \\
  Android + Embarcado &  Dados móveis &                    3,28 &                    3,16 &                    3,25  \\
  Android + Embarcado &         WI-FI &                    2,72 &                    2,68 &                    2,78  \\
\bottomrule
\end{tabular*}
\end{table}


\section{SISTEMA EMBARCADO}
O sistema embarcado foi testado quanto à detecção de acidentes em eventos de queda e rotação, bem como em sua comunicação com o dispositivo \textit{Android}. Também foi observado seu consumo de energia como forma de quantizar o tempo de operação nos casos em que o dispositivo não esteja sendo alimentado pelo veículo.

\subsection{\textbf{Quedas}}
O teste de quedas foi realizado a fim de medir a força dos impactos de forma a obter as faixas sensíveis a cada velocidade de impacto. Para os testes foram realizadas três quedas consecutivas de 0.5m, 1m, 1.5m e 2m para cada faixa de sensibilidade pretendida. O teste foi realizado em dois tipos de superfícies sendo a macia (amortecido por colchonete de 3 cm) e rígida (queda sobre o piso). 

Utilizando a equação de Torricelli adaptada para quedas, é possível calcular a velocidade de impacto com base na altura. Ao total foram realizadas 72 quedas. A média obtida entre os valores é demonstrado nas tabelas 7 e 8.

\begin{table}[H]
    \centering
    \caption{Tabela valores médios do giroscopio e acelerometro em superfície macia}
    \begin{tabular*}{\textwidth}{l@{\extracolsep{\fill}}lllllll}
\toprule
{} &                 Altura  &       Velocidade Estimada (km/h) & Queda 1(g) & Queda 2(g) & Queda 3(g) \\
\midrule
0 &              0,5     &   11,28      &        3,522 &                    3,627 &                    3,684  \\
1 &              1 &         15,95 &                    3,947 &                    4,176 &                     3,967  \\
2 &  1,5 &  19,53 &                    4,425 &                    4,39 &                    4,573 \\
3 &  2 &         22,55 &                    4,719 &                    4,759 &                    4,721  \\
\bottomrule
\end{tabular*}


\end{table}


\begin{table}[H]
    \centering
    \caption{Tabela valores médios do giroscopio e acelerometro em superfície rígida}
    \begin{tabular*}{\textwidth}{l@{\extracolsep{\fill}}lllllll}
\toprule
{} &                 Altura  &       Velocidade Estimada (km/h) & Queda 1(g) & Queda 2(g) & Queda 3(g) \\
\midrule
0 &              0,5     &   11,28      &        3,549 &                    3,654 &                    3,632  \\
1 &              1 &         15,95 &                    4,004 &                    4,096 &                     4,044  \\
2 &  1,5 &  19,53 &                    4,551 &                    4,501 &                    4,534 \\
3 &  2 &         22,55 &                    4,911 &                    4,967 &                    4,919  \\
\bottomrule
\end{tabular*}

\end{table}

Os valores para o teste de queda mensurados para ambas superfícies são próximos, porém é possível perceber um aumento da força registrada na superfície rígida em relação a macia conforme é aumentada a altura. Essa maior proximidade dos valores registrados em menores alturas é devido a baixa velocidade a qual o sistema embarcado está sujeito no momento do impacto.


\subsection{\textbf{Rotação}}
O teste de rotação compreende a rotação do dispositivo embarcado de forma a detectar os acidentes por tombamento da motocicleta. Nele foram comparados os ângulos obtidos pelo sistema embarcado com um transferidor, de forma a identificar possíveis variações em suas leituras. Para cada ângulo real o valor foi amostrado três vezes. O resultado dessa análise pode ser visto na tabela 9.


\begin{table}[H]
    \centering
    \caption{Teste de rotação do sistema embarcado}
    \begin{tabular*}{\textwidth}{l@{\extracolsep{\fill}}lllllll}
\toprule
{} &                 Eixo  &       Ângulo esperado (º) & valor 1(º) & valor 2(º) & valor 3(º) \\
\midrule
0 &              X     &   50      &        54 &                    53,7 &                    54,2  \\
1 &              Y &         50 &                    56 &                    55,3 &                     55,6  \\

2 &              X &         60 &                    63,5 &                    63,8 &                     64  \\

3 &              Y &         60 &                    64,8 &                    64,2 &                     64,7  \\

4 &              X &         70 &                    74,3 &                    74,1 &                     73,9  \\

5 &              Y &         70 &                    75,5 &                    75,2 &                     75,7  \\




\bottomrule
\end{tabular*}

\end{table}


A diferença entre o esperado e o mensurado obtido na rotação se deve à posição em que o módulo MPU está posicionado, pois ele se encontra em um pequeno desnível em relação ao ângulo esperado. Esse desnível se deve a natureza de construção do protótipo, sendo seu posicionamento na placa o maior agravante, mas também se deve ao posicionamento e pressão da espuma do preenchimento que pode deslocar-se ligeiramente com os movimentos do veículo.



\subsection{\textbf{Conectividade}}
Foram realizados testes relacionados à conectividade entre o sistema embarcado e o aplicativo \textit{Android}. O teste consistiu no afastamento dos dispositivos com posterior aproximação em 15 tentativas distintas, observando-se assim a reconectividade entre os aparelhos. Nesse teste foram observados os eventos de reconexão bem sucedida, reconexão manual (foi realizado logout e login no aplicativo) e problema crítico (foi necessário realizar um novo pareamento no dispositivo). Os resultados seguem na tabela 10.

\begin{table}[H]
    \centering
    \caption{Teste de conexão entre o sistema embarcado e aplicativo Android}
    \begin{tabular*}{\textwidth}{l@{\extracolsep{\fill}}lllllll}
\toprule
{} &                 Evento  &       Quantidade  & Taxa de reconexão \\
\midrule
0 &              Desconexões     &   15      &       -   \\
1 &              Reconexão automática &         11 &                    73,33\% \\

2 &              Reconexão manual &         3 &                    26,66\%  \\

3 &              Problema crítico &         1 &                    6,66\%   \\

\bottomrule
\end{tabular*}
\end{table}

O problema crítico foi investigado buscando encontrar sua causa. Após testes foi diagnosticado que o problema ocorreu por alguma falha de pareamento onde o embarcado demonstrava que sua conexão estava ativa e funcional, porém o \textit{smartphone} não conseguiu restabeler a conexão. 


\subsection{\textbf{Simulação de acidentes}}

Ao analisar os resultados obtidos nos testes de rotação e queda foi estipulado que a sensibilidade ideal é de 4,3g para a detecção de impactos, o equivalente à detecção de acidentes com velocidade aproximada de 20km/h, e 65º para a rotação. Com estes valores parametrizados foram realizadas mais 20 quedas em cada superfície comentada anteriormente nas alturas de 1m e 1.5m de forma a validar sua detecção.

Das 20 quedas realizadas para cada superfície, foram separadas 10 para cada altura alteriormente comentada. Desse modo as 10 quedas para cada altura foram realizadas sequencialmente, sem interrupções ou reinício de qualquer parte do sistema, com o acompanhamento da detecção de acidentes em tempo real. Os resultados são apresentados na tabela 11.



\begin{table}[H]
\caption{Simulação de acidente com acelerômetro}
\begin{adjustbox}{width=\columnwidth,center}
\begin{tabular}{lllllll}
\toprule
{} &        Superfície     &    Altura (metros)  &       Quantidade & Acerto & falso positivo & falso negativo \\
\midrule
0 &              Macia     &    1         &    10  &  10   &  0 &  0  \\
1 &              Macia     &    1.5       &    10  &  9   &  0 &  1  \\
2 &              Rígida    &    1         &    10  &  8   &  2 &  0  \\
3 &              Rígida    &    1.5       &    10  &  10  &  0 &  0  \\

\bottomrule
\end{tabular}

\end{adjustbox}
\end{table}


Os valores de falso positivo e falso negativo mensurados nos testes foram muito próximos do limiar estipulado. A obtenção desses resultados pode ser atribuída a região de impacto da caixa do sistema embarcado com a superfície, podendo atingir pontos mais macios ou mais rígidos, o que afeta o valor mensurado.

\section{CONSUMO}

\subsection{\textbf{Aplicativo}}

Para o consumo de bateria do \textit{smartphone} foram adotas duas formas de testes. A primeira verifica o consumo da bateria sem a utilização do \textit{Bluetooth} em um período de 8 horas e outro teste conectado com o \textit{Bluetooth} no sistema embarcado no mesmo período de tempo. Para os testes foi considerado o consumo do aplicativo e não o consumo dos módulos habilitados no \textit{Android}.

Para o primeiro teste, considerando que ambos os sensores estavam em operação juntamente com o GPS e rede \textit{WiFi}, foi encontrado um valor de 15 mAh.
Como o teste foi realizado com um \textit{Zenfone} 4 ZE554XL que possui 3300 mAh de autonomia o valor encontrado representa  0,45\% da bateria no período de uma  hora.

Já no segundo teste utilizando novamente GPS,\textit{Wifi} e os sensores com a adição do \textit{Bluetooth} se comunicando com a aplicação, foi encontrado o valor médio de  27 mAh no aplicativo gerando um consumo de 0,81\% do consumo da bateria. A Figura 36 demostra de forma visual o consumo do aparelho para os dois casos.


 \begin{figure}[H]

\begin{center}
     \caption{Consumo de bateria do aplicativo}
  \includegraphics[width=150mm]{images/Cap5/consumo_grafico.png}
\end{center}
 \scriptsize Fonte: Os autores
  
\end{figure}


Analisando os dois casos e considerando uma rotina de  trabalho de um entregador de oito horas sem o \textit{Bluetooth} conectado teria um consumo teórico de 3,6\% e com o \textit{Bluetooth} conectado de  6,48\% sendo assim  um consumo relativamente, baixo sem considerar que muitos motoboys utilizam de power banks para recarregar os celulares enquanto se locomovem pela cidade.









\subsection{\textbf{Sistema embarcado}}

A fim de validar o consumo do sistema embarcado, foi realizada uma pesquisa com o consumo teórico do mesmo, bem como o total real mensurado. Para o módulo MPU-6050 o consumo teórico é de 3.9 mA operando com 6 eixos digitais mais DMP, enquanto para a ESP-32 o consumo teórico de seu processador operando em 240 MHz é de 73 mA, porém com a adição do Bluetooth seu consumo se eleva para 141 mA. O total teórico desconsiderando as perdas do regulador de tensão é de aproximadamente 145 mA, enquanto o valor real mensurado é de 154 mA. A mensuração do valor real foi obtida utilizando um multímetro em série entre o sistema embarcado e uma bateria de moto, levando em consideração o regulador de tensão presente.

Estes valores pesquisados e mensurados para o sistema embarcado foram obtidos apenas por fins de curiosidade, uma vez que o posicionamento do sistema embarcado é no veículo e tem sua alimentação compartilhada com o mesmo.





\section{Conclusão}

Este trabalho apresenta um sistema de monitoramento de acidentes de moto via aplicativo mobile com a extensão de um sistema embarcado. O sistema aposta na política da "boa vizinhança" entre os usuários para seu funcionamento e é responsável por identificar e notificar os usuários de acidentes ocorridos dentro de um raio delimitado. Também é possível adquirir dados relevantes sobre os acidentes como sua data, hora, localização e situação climática.

Com o objetivo de reduzir o tempo de resposta ao prestar os primeiros socorros, o sistema ainda possibilita a visualização dos dados anônimos por parte do público e órgãos competentes permitindo não só uma resposta mais rápida e acertiva quanto aos acidentes motociclisticos como permite a implantação de políticas preventivas em regiões mais perigosas.

Durante a fase de testes foram realizadas avaliações que envolvem todo o ciclo de vida do sistema, compreendendo a criação de um novo usuário e a simulação de acidentes com um ou mais dispositivos dentro do raio determinado. A simulação consistiu em quedas controladas do \textit{smartphone} e do sistema embarcado em momentos distindos, ocasionando a correta ação de notificação de todos os usuário ativos dentro da área delimitada pelo raio traçado a partir do acidentado. No momento do acidente simulado também foi observada a atualização em tempo real dos gráficos e relatórios presentes na plataforma Web.

Ao testar os componentes necessários para o sistema em ambiente controlado, foi possível comprovar a eficácia do sistema em detectar acidentes e notificar todos os usuário próximos em um raio pré-determinado, bem como analisar a rapidez de todos os recursos que necessitam de internet em diferentes cenários, possibilitando assim verificar que os objetivos propostos ao longo deste trabalho foram alcançados.

A plataforma Web contendo relatórios possibilita uma simples visualização dos dados sobre acidentes, permitindo que outros serviços possam utilizá-los. Nesse sentido o sistema prova-se uma alternativa eficiente não só em reduzir o tempo de resposta para os primeiros socorros em casos se acidente, mas também arquiva e disponibiliza dados que podem ser utilizados em políticas de prevenção e assistência.




\clearpage

\section{Trabalhos Futuros}

Como possíveis trabalhos futuros, pode-se apontar:

\begin{enumerate}
    \item Enriquecer os dados adquiridos com as informações do sensores para criar uma base para análise preditiva de acidentes em regiões específicas, horários ou tipo de clima.
    
    \item Integrar o sistema com os órgãos  públicos gerando análises de comportamento de trânsito, regiões perigosas por hora e clima sendo possível verificar a velocidade do tráfego por região aumentando a fiscalização ou ajustando os limites das vias.
    
    \item Para empresas privadas é possível realizar todo o rastreio da frota de veículos de entrega, sendo capaz de verificar o desempenho de cada usuário em situações diferenciadas como em dia chuvosos, nublados e além de poder fazer todo um rastreio da rota realizada por usuário sendo capaz de realizar otimização de rotas de entregas e distribuição das entregas para os usuários com mais aptidão em tipos de climas ou horários especificos.
    
\end{enumerate}










    \chapter{Conclusão}\label{cap:conclusão}

Neste trabalho foi estudada a falta de recursos tecnológicos de baixo custo e fácil utilização para educação básica. Além disso, pensando nas profissões do futuro, foi estudada a importância do ensino de lógica de programação para as crianças de hoje (2020).
Pensando nesses problemas apresentados, foi proposto e desenvolvido um aplicativo que tem o objetivo de tentar ensinar programação para crianças do ensino básico e fundamental 1. 

O aplicativo é um jogo que interage com blocos físicos por meio de reconhecimento de imagem. O aplicativo consiste em um jogo com o tema de reciclagem no qual por meio de blocos físicos, com instruções de andar, virar, esperar, repetir e blocos numerados de 0 a 9, a criança deve resolver desafios de lógica proposto pelo jogo organizando esses blocos em sequências lógicas, fotografando e submento pelo aplicativo para validar se suas instruções estão corretas ou não. O aplicativo foi desenvolvido em Unity e Python3 e os blocos foram impressos por uma impressora 3D e coloridos manualmente com tinta. O resultado final do trabalho atendeu à todos os objetivos propostos desde o desenvolvimento do aplicativo jogo e dos blocos físicos até a realização de testes com crianças.

Após o término do desenvolvimento do aplicativo jogo, foram realizados testes de funcionamento e testes com crianças. O objetivo destes testes foram analisar o funcionamento e a capacidade do aplicativo jogo em reter a atenção da crianças, capacidade de ensinar conceitos básicos de programação e competência de ensinar sobre reciclagem.

Durante o desenvolvimento do aplicativo jogo, foram identificadas algumas possibilidades de otimização para este trabalho, como por exemplo a escolha das cores dos blocos. Por meio dos testes práticos, foi observado que a cor amarelo não se comporta bem na etapa de reconhecimento de imagem pois quando é exposta diretamente a luz tem seus parâmetros de RGB facilmente distorcidos.

Após a conclusão do trabalho e realização dos testes, foram identificadas possibilidades de trabalhos futuros. Como possíveis trabalhos futuros, pode-se apontar:

\begin{enumerate}
    \item Desenvolvimento do aplicativo para outros sistemas operacionais, como o iOS. 
    
    \item Desenvolvimento de novos tipos de blocos.
    
    \item Desenvolvimento de novas fases que utilizem mais blocos ou até mesmo outros tipos de blocos.
    
    \item  Refinar o algoritmo de reconhecimento dos blocos para reconhecer blocos desenvolvidos com outros materiais, desta forma, seria possível disponibilizar um pdf com instruções de montagem de blocos e desta forma a própria criança, por meio desse passo a passo, poderia construir seus blocos com papel, deixando assim a solução ainda mais econômica.

    
\end{enumerate}










    
    \phantompart

    \postextual
    \bibliography{Bibliografia}

    \includepdf[pages=-,scale=.8,pagecommand={APÊNDICE A - PARECER CONSUBSTANCIADO DO CEP}\label{appendix:apendice_a},linktodoc=true]{Apendice/parecer_comite.pdf}

\includepdf[pages=-,scale=.8,pagecommand={APÊNDICE B - TERMO DE CONSENTIMENTO LIVRE E ESCLARECIDO}\label{appendix:apendice_b},linktodoc=true]{Apendice/TCLEFase2.pdf}

\includepdf[pages=-,scale=.8,pagecommand={APÊNDICE C - TERMO DE ASSENTIMENTO LIVRE E ESCLARECIDO}\label{appendix:apendice_c},linktodoc=true]{Apendice/TermoDeAssentimento_v2.pdf}




    \includepdf[pages=-,scale=.8,pagecommand={ANEXO A - PARECER CONSUBSTANCIADO DO CEP}\label{appendix:apendice_a},linktodoc=true]{Apendice/parecer_comite.pdf}

\end{document}