\documentclass{utfpr-pg}

\usepackage{cmap}
\usepackage[T1]{fontenc}
\usepackage{graphicx}
\usepackage{latexsym}
\usepackage{amssymb}
\usepackage{lipsum}
\usepackage{pdfpages}
\usepackage{float}
\usepackage{color,soul}
\usepackage{xcolor}
\usepackage{adjustbox}
\usepackage{quoting}
\usepackage{tocloft}
\usepackage[font=small,labelfont=bf]{caption}

\cftsetindents{section}{0em}{1.2cm}
\cftsetindents{subsection}{0em}{0.85cm}
\cftsetindents{subsubsection}{0cm}{0.65cm}
\cftsetindents{chapter}{0pt}{1.7cm}

\makeatletter

\renewcommand*\l@chapter{\@dottedtocline{0}{0em}{1.5cm}}
\makeatother

\graphicspath{{Imagens/}}

\DeclareFloatingEnvironment[
fileext=lod,
listname=Lista de Definições,
name=Definição,
placement=tbhp,
]{definicao}


\curso{}
\autor{ Alana Mafra Netto  \\  Everton Henrique Carneiro }
\titulo{Aplicativo para ensino de lógica de programação utilizando visão computacional}
\local{Curitiba}
\data{2020}
\preambulo{Monografia apresentada como requisito parcial de aprovação da disciplina de Trabalho de Conclusão de Curso de Engenharia da Computação da Universidade.\break}

\orientador{Dra Veronica Isabela Quandt}

% informações do PDF
\makeatletter
\hypersetup{
        % pagebackref=true,
        pdftitle={\@title},
        pdfauthor={\@author},
        pdfsubject={\imprimirpreambulo},
        pdfcreator={LaTeX with abnTeX2},
        colorlinks=false,
}
\makeatother

% Controle do espaçamento entre um parágrafo e outro:
\setlength{\parskip}{0.1cm}  % tente também \onelineskip

\makeindex
\begin{document}
% Retira espaço extra obsoleto entre as frases.
\frenchspacing

\imprimircapa
\imprimirfolhaderosto

% % Termo de aprovação
% \includepdf[pages=-]{images/termo.pdf}

% % Agradecimentos
% \input{Textos/Agradecimentos}

% % Resumo
% \begin{resumo}
\refthis{bibli}

\textbf{Palavras-chaves}: Lógica de Programação, Educação, Crianças, Jogo, Blocos
\end{resumo}

\begin{resumo}[Abstract]
\refthis[en]{bibli}

\textbf{Key-Words}: Programming Logic,  Education, Children, Game, Blocs
\end{resumo}

% % Lista de Figuras
% \listoffigures*
% \clearpage

% % Lista de tabelas
% \listoftables*
% \clearpage

% Sumário
\tableofcontents*
\clearpage

\textual
\pagestyle{simple}

\captionsetup{singlelinecheck = false, justification=raggedright, labelsep=space}

% entre chaves, adicionar o nome do capítulo.
% o comando \label cria um nome para o capítulo que pode ser utilizado com referência cruzada ao longo do texto.

\chapter{Introdução}\label{intro}

A tecnologia tem ganhado cada vez mais relevância em diversas áreas como saúde, indústria, agricultura, entre outros. Um dos campos fortemente impactado pela tecnologia é a Educação. Isso ocorre por conta das facilidades que a tecnologia proporciona em relação ao acesso à informação e pelas possibilidades de novas metodologias de ensino. 


\section{Problema}

Apesar de todos os benefícios que a tecnologia trás para a educação, poucos professores fazem uso de novas tecnologias. Conforme a pesquisa do Programa Nacional de Formação Continuada em Tecnologia Educacional \cite{suenia_andre_2012} , na maioria dos casos, a utilização da tecnologia fica restrita somente a laboratórios de informática. Esse resultado decorre da insegurança desses profissionais na utilização da tecnologia no cotidiano. Observa-se ainda, de acordo com a pesquisa, que a introdução de novas tecnologias nas escolas apresenta grande carência de investimentos.

Portanto, pode-se inferir que existe uma demanda de tecnologias acessíveis financeiramente. Para que os professores possam utilizá-las de forma fácil e intuitiva nos ambientes escolares. Há também uma necessidade desse aprendizado por parte dos alunos, visto que inúmeras profissões estão exigindo um novo conhecimento tecnológico no mercado de trabalho, como a utilização da lógica de programação. Outro ponto a ser citado é a necessidade de inserção de jogos eletrônicos como agentes de ensino, devido a grande motivação e aderência dos alunos com esse tipo de abordagem \cite{kaue_tatiane_marcos_2017}.

Assim como tecnologia, qualidade de vida é um tema que vem ganhando mais relevância no nosso cotidiano. Soluções inovadoras e tecnológicas são apresentadas para tentar salvar o mundo dos problemas criados pela humanidade. A educação ambiental é responsável por formar pessoas mais preocupadas com problemas ambientais, que busquem a conservação e preservação do Meio Ambiente.

No ano de 2017, cerca de 42\% dos resíduos do Brasil não foram destinados corretamente, ou seja, por volta de 30 milhões de toneladas \cite{abrelpe_2017}. Pode-se afirmar que tal ação não é benéfica para o meio ambiente, já que esse material pode acabar em rios, gerar enchentes e causar impactos na saúde pública. Da totalidade de lixo produzido em território brasileiro, cerca de 30\% poderia ser reciclado, entretanto somente 3\% disso é realmente reciclado \cite{}. Um dos principais motivos disso é o fato do lixo orgânico e reciclável não serem descartados corretamente. Desse modo, reciclar é uma ótima alternativa para problemáticas de resíduos urbanos, impactando diretamente o meio ambiente, sociedade e economia. 

\section{Justificativa}

Com a evolução dos perfis de trabalho, necessitando cada vez mais de conhecimentos básicos de programação, aumenta a procura de tecnologias no ambiente educacional, possibilitando a inclusão de novos métodos de ensino utilizando jogos digitais, realidade aumentada, simuladores e etc.

A qualidade de vida, junto com sustentabilidade, são temas que, ano após ano, vem ganhando mais força, nos fazendo pensar e agir de forma a inclui-los no nosso cotidiano através de reciclagem, práticas para diminuir a produção de lixo e reduzir o impacto ambiental.

Com isso em vista, fica clara a necessidade de ensinar para as crianças a importância da reciclagem e principalmente o descarte correto do lixo de uma maneira simples didática, aproveitando tudo o que a tecnologia tem a oferecer.

Para possibilitar essa ação, este trabalho propõe um jogo, cujo objetivo é fazer o descarte correto de lixo, Objetivo de Desenvolvimento Sustentável da ONU número 12.5 \cite{onu_2015}, utilizando a lógica de programação com blocos físicos.

\section{Objetivo Geral}

Desenvolver um aplicativo capaz de auxiliar o professor no ensino da lógica de programação para crianças em idade escolar, utilizando como tema a reciclagem. No jogo o usuário deverá descartar cada tipo de lixo em sua respectiva lixeira, para tal serão utilizados blocos físicos, que serão reconhecidos por meio da aplicação de visão computacional, para compor a lógica que permitirá direcionar o personagem do jogo para percorrer o caminho correto para a lixeira.

\section{Objetivos Específicos}

\begin{enumerate}
\item Construir os blocos físicos de acordo com o padrão definido.
\item Desenvolver o jogo preparado para a captura e envio da imagem dos blocos.
\item Realizar a interpretação da imagem adquirida durante o jogo utilizando visão computacional.
\item Desenvolver a integração entre as ações do jogo e o código retornado pelo servidor.
\item Construir banco para armazenar informações do jogo e popular um \textit{dashboard} para análises estatísticas.
\item Realizar testes com um grupo de crianças da faixa etária determinada para validação da efetividade do sistema.
\end{enumerate}

\chapter{Revisão Bibliográfica} \label{cap:rev}
Este capítulo tem como objetivo apresentar as bases teóricas que apóiam e norteiam o desenvolvimento do projeto e analisar outros trabalhos relacionados ao tema para ilustrar e proporcionar uma melhor compreensão para o leitor.

\section{Fundamentação teórica} 
Frequentemente são publicados novos estudos especulativos sobre as transformações que ocorrerão no mercado de trabalho no futuro. Apesar das análises e hipóteses variarem, a maioria aponta que diversas profissões de hoje ficarão obsoletas e em paralelo a isso, diversas outras novas profissões nascerão. No artigo \textit{21 Jobs Of The Future feito pela Cognizant Technology Solutions} \cite{cognizant_2017} vinte e uma novas profissões e suas principais habilidades são apontadas, servindo como um guia para conseguir um emprego ou se manter no mercado de trabalho nos próximos dez anos. Na maioria dessas novas profissões inferidas por esses e outros artigos, a habilidade de programar é aplicável direta e indiretamente. Algumas dessas profissões são: desenvolvedores de \textit{softwares}, engenheiros de \textit{machine learning}, analista de cibersegurança, engenheiros de \textit{big data}, cientista de dados, entre outras.

David Baker é um escritor, jornalista, fundador da TSOL Brasil, co-fundador da revista \textit{Wired} e um dos membros com mais antigos do corpo docente da \textit{The School of Life} de Londres. No ano de 2015 em uma palestra em São Paulo, Baker começa seu discurso com a seguinte afirmação: “O seu emprego pode não existir amanhã” \cite{carvalho_2015}. David Baker é bastante conhecido pelas suas pesquisas sobre as relação da tecnologia com o mercado de trabalho e afirma acreditar fortemente que logo grande parte das carreiras de hoje serão substituídas por robôs. Segundo Baker, não só carreiras braçais serão substituídas por robôs, mas “os engravatados também estão ameaçados”, brinca Baker.

Hoje, em 2020, já é possível ver essa migração. Indústrias de todos os tamanho, estão substituindo seus trabalhadores por robôs, um exemplo disso é a empresa de \textit{e-commerce} Amazon e sua logística interna, operada quase 100\% por tais máquinas \cite{winick_2018}. No mercado financeiro é possível observar Inteligências Artificiais atuando na compra e venda de ações de forma automatizada, com mais eficiência e assertividade do que analista financeiros. Na área da saúde, nano robôs já estão fazendo cirurgias. No setor de mobilidade urbana existem transportes completamente autônomos, como os caminhões sem motorista da empresa de transporte Uber que já transportam cargas sozinhos em rodovias americanas \cite{demartini_2016}. Até mesmo em trabalhos que exigem habilidades criativas, podemos ver robôs atuando e um exemplo disso é o comercial de uma marca de veículos de luxo escrito totalmente por uma inteligência artificial \cite{autran_2018}.

Sendo assim, torna-se evidente a importância de preparar as crianças de hoje para o mercado de trabalho do futuro. Para tal desenvolvimento, uma das habilidades mais importantes é a programação. Um exemplo dessa relevância foi a recente atualização da Base Nacional Comum Curricular (BNCC) realizada pelo Ministério da Educação (MEC). Nessa atualização, a BNCC dedica uma de suas dez competências para as tecnologias digitais no seu conceito de educação integral. Segundo a BNCC:

\begin{citacao}

Compreender, utilizar e criar tecnologias digitais de informação e comunicação de forma crítica, significativa, reflexiva e ética nas diversas práticas sociais (incluindo as escolares) para se comunicar, acessar e disseminar informações, produzir conhecimentos, resolver problemas e exercer protagonismo e autoria na vida pessoal e coletiva \cite[p. 9]{bncc_2017}.

\end{citacao}

Analisando esse trecho da Base Nacional Comum Curricular, é possível reconhecer argumentos em prol da inserção de programação na Educação Básica, como ao apontar que tecnologias digitais podem ser um excelente recurso para a comunicação de informações e resolução de problemas. A BNCC também menciona o termo “pensamento computacional” no Caderno de Matemática, comentando a importância de fluxogramas e algoritmos, que podem ser estudados nas aulas Matemática.

Em um estudo sobre ensino inicial de Programação e Robótica Educacional \cite{antonello_cardoso_2015} apontam-se que o ensino de programação pode ser interdisciplinar, ou seja, pode abranger duas ou mais áreas de conhecimento. O ensino de programação pode proporcionar interação e progresso em duas disciplinas ao mesmo tempo, desenvolvendo \textit{hard skills}, entendidas como competências técnicas. Do mesmo modo, esse ensino está diretamente ligado ao desenvolvimento de \textit{soft skills}, que são habilidades que trabalham com a relação dos indivíduos com os outros e com eles mesmos. Essas são competências como: resiliência, colaboração e comunicação, afirma o autor do livro best seller “Inteligência Emocional” \cite{goleman_2012}.

O artigo “Programar é bom para as crianças? Uma visão crítica sobre o ensino de programação nas escolas” \cite{geraldes_2014} mostra — por meio do Scratch, que é uma ferramenta de ensino de programação em blocos para crianças — que o ensino da programação desenvolve competências como criatividade e raciocínio lógico, além de estimular o aprendizado de inglês, trabalho em equipe, resolução de problemas, entre outras.

Mesmo com todos esses benefícios, ainda é pequeno o número de professores que fazem realmente o uso de tecnologias ou ensinam programação e outras tecnologias em sala de aula, os que usam, fazem somente uso de laboratórios de informática, como diz o estudo do Programa Nacional de Formação Continuada em Tecnologia Educacional \cite{suenia_andre_2012}. Esse estudo mostra que professores têm insegurança ao utilizar tecnologias em sala de aula, observa-se retratado no mesmo estudo que a inserção de novas tecnologias nas escolas sofre de falta de investimentos também. Portanto, nota-se que há uma carência de tecnologias que sejam acessíveis financeiramente e que os professores possam utilizá-las facilmente e intuitivamente em sala de aula. Também existem necessidades por parte dos alunos.

Em um artigo apresentado no Congresso Internacional de Educação e Tecnologias \cite{lima_queiroz_santana_2018}, é apresentado os seguintes estilos de aprendizagem: visual, auditivo e cinestésico. Assim, também é retratada a dificuldade do aluno de hoje em se concentrar em aulas realizadas com os métodos tradicionais e antiquados. Esse estudo ressalta a importância do uso de TIDCs (Tecnologias Digitais de Informação e Comunicação) em sala de aula, pois, segundo o estudo, TIDCs conseguem alinhar os estilos de aprendizado além de aumentar a concentração e motivação dos alunos. É por esse motivo, que na sala de aula se faz necessário observar a dinâmica do dia a dia, assim como as particularidades dos alunos, como idade, região em que vivem, interesses, etc. De acordo com o professor húngaro \cite[p. 81]{dornyei_2001} existem algumas estratégias motivacionais que demonstram efetividade no ensino, são elas: o aumento da interação dos alunos, a atribuição de tarefas interessantes e a quebra da monotonia da aprendizagem. Logo, se trabalhadas de forma interligada, essas estratégias podem tornar as aulas mais instigantes e estimular o desejo de aprender nos alunos.

Ademais, habilidades tecnológicas se tornam cada vez mais requisitadas no mercado de trabalho, reforçando assim a importância do contato do aluno com esses conceitos e ferramentas. Tendo em vista essa dificuldade de concentração dos alunos, é necessário criar estratégias que estimulem o aprendizado dos alunos. \cite{jacobsen_maffei_sperotto_2013}, sugerem o uso de jogos eletrônicos, pois por serem lúdicos, os jogos tornam a aprendizagem mais eficiente. Já que estimulam o raciocínio rápido, auxiliam na assimilação de conceitos complexos e desenvolvem a criatividade, isso porque com jogos os alunos deixam de ser ouvintes e passam a ser protagonistas do seu aprendizado. Nesse mesmo artigo, “Jogos eletrônicos: um artefato tecnológico para o ensino e para a aprendizagem”, nota-se que jogos não auxiliam apenas no aprendizado do conteúdo abordado em um determinado jogo, mas estimulam o desenvolvimento de outras competências como convivência, cooperação, troca de ideias, cumprimento de regras, entre outros hábitos de interação. Assim, acredita-se que a inserção de jogos eletrônicos pode funcionar como motivação para os alunos.

Outro ponto considerado importante no âmbito da educação é o ensino da sustentabilidade. No ano de 2016, cerca de 40\% dos resíduos sólidos não foram destinados corretamente, ou seja, por volta de 30 milhões de toneladas \cite{abrelpe_2017}. Pode-se afirmar que tal ação não é benéfica para o meio ambiente, já que esse material pode acabar em rios, gerar enchentes e causar impactos na saúde pública. Da totalidade de lixo produzido em território brasileiro, cerca de 30\% poderia ser reciclado, entretanto somente 3\% disso é realmente reciclado \cite{pnrs_2010}. Um dos principais motivos disso é o fato do lixo orgânico e reciclável não serem descartados corretamente. Desse modo, reciclar é uma ótima alternativa para amenizar a problemática de resíduos urbanos, impactando diretamente o meio ambiente, sociedade e economia.

De acordo com a “Agenda 2030 para o Desenvolvimento Sustentável”, a ONU (Organizações das Nações Unidas) tem como objetivo reduzir consideravelmente os resíduos até o ano de 2030 \cite{onu30_2015}. A ideia é fazer isso por meio prevenção, da reciclagem e do reuso. Tal ação pode ser desenvolvida por meio da Educação Ambiental. Essa estratégia possibilita a sensibilização dos alunos em relação ao meio ambiente, instruindo-os a refletir sobre a poluição e os danos causados à natureza. Sobre esse tema os PCNS (Parâmetros
Curriculares Nacionais) apontam que:

\begin{citacao}

O trabalho com o tema Meio Ambiente deve ser desenvolvido visando-se proporcionar aos alunos uma diversidade de experiências e ensinar-lhes formas de participação, para que possam ampliar a consciência sobre as questões relativas ao meio ambiente e assumirem de forma independente e autônoma atitudes e valores voltados à sua proteção e melhoria \cite[p. 46]{pcns_2001}.

\end{citacao}


Sendo assim, ao desenvolver esse tema no ambiente escolar, pode-se ensinar aos estudantes o respeito à natureza e o cuidado com o meio ambiente. Um bom jeito de trabalhar esse tema é por meio da reciclagem, ensinando as crianças sobre os benefícios que esse ato pode trazer para o meio ambiente e para eles mesmos.

Por fim, considerando os conceitos apresentados previamente é possível julgar como relevante trabalhar com o uso de programação e de jogos nas escolas. Já que essas têm capacidade de abranger diversas áreas do conhecimento, além de poder desenvolver habilidades emocionais. Assim, esse projeto visa unir o ensino de programação por meio de jogos com o tema de reciclagem, a fim de criar uma excelente ferramenta para o ensino dessas áreas nas escolas.

\section{Trabalhos Relacionados}


\clearpage

\addcontentsline{toc}{chapter}{REFÊRENCIAS}
\bibliography{bibli}

% \newpage

\end{document}